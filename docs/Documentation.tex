%% Generated by Sphinx.
\def\sphinxdocclass{report}
\documentclass[letterpaper,10pt,english]{sphinxmanual}
\ifdefined\pdfpxdimen
   \let\sphinxpxdimen\pdfpxdimen\else\newdimen\sphinxpxdimen
\fi \sphinxpxdimen=.75bp\relax
\ifdefined\pdfimageresolution
    \pdfimageresolution= \numexpr \dimexpr1in\relax/\sphinxpxdimen\relax
\fi
%% let collapsible pdf bookmarks panel have high depth per default
\PassOptionsToPackage{bookmarksdepth=5}{hyperref}

\PassOptionsToPackage{booktabs}{sphinx}
\PassOptionsToPackage{colorrows}{sphinx}

\PassOptionsToPackage{warn}{textcomp}
\usepackage[utf8]{inputenc}
\ifdefined\DeclareUnicodeCharacter
% support both utf8 and utf8x syntaxes
  \ifdefined\DeclareUnicodeCharacterAsOptional
    \def\sphinxDUC#1{\DeclareUnicodeCharacter{"#1}}
  \else
    \let\sphinxDUC\DeclareUnicodeCharacter
  \fi
  \sphinxDUC{00A0}{\nobreakspace}
  \sphinxDUC{2500}{\sphinxunichar{2500}}
  \sphinxDUC{2502}{\sphinxunichar{2502}}
  \sphinxDUC{2514}{\sphinxunichar{2514}}
  \sphinxDUC{251C}{\sphinxunichar{251C}}
  \sphinxDUC{2572}{\textbackslash}
\fi
\usepackage{cmap}
\usepackage[T1]{fontenc}
\usepackage{amsmath,amssymb,amstext}
\usepackage{babel}



\usepackage{tgtermes}
\usepackage{tgheros}
\renewcommand{\ttdefault}{txtt}



\usepackage[Bjarne]{fncychap}
\usepackage[,numfigreset=2,mathnumfig,mathnumsep={.}]{sphinx}

\fvset{fontsize=auto}
\usepackage{geometry}


% Include hyperref last.
\usepackage{hyperref}
% Fix anchor placement for figures with captions.
\usepackage{hypcap}% it must be loaded after hyperref.
% Set up styles of URL: it should be placed after hyperref.
\urlstyle{same}

\addto\captionsenglish{\renewcommand{\contentsname}{Contents:}}

\usepackage{sphinxmessages}
\setcounter{tocdepth}{3}
\setcounter{secnumdepth}{3}


\title{GloMarGridding}
\date{Feb 21, 2025}
\release{0.2.0}
\author{NOC Surface Processes}
\newcommand{\sphinxlogo}{\vbox{}}
\renewcommand{\releasename}{Release}
\makeindex
\begin{document}

\ifdefined\shorthandoff
  \ifnum\catcode`\=\string=\active\shorthandoff{=}\fi
  \ifnum\catcode`\"=\active\shorthandoff{"}\fi
\fi

\pagestyle{empty}
\sphinxmaketitle
\pagestyle{plain}
\sphinxtableofcontents
\pagestyle{normal}
\phantomsection\label{\detokenize{index::doc}}


\sphinxstepscope


\chapter{Introduction}
\label{\detokenize{introduction:introduction}}\label{\detokenize{introduction::doc}}
\sphinxstepscope


\chapter{Getting Started}
\label{\detokenize{getting_started:getting-started}}\label{\detokenize{getting_started::doc}}

\section{Installation}
\label{\detokenize{getting_started:installation}}

\subsection{Via Pip}
\label{\detokenize{getting_started:via-pip}}
\sphinxAtStartPar
GloMarGridding is not available on PyPI, however it can be installed via pip with the following command:

\begin{sphinxVerbatim}[commandchars=\\\{\}]
\PYG{g+go}{pip install GloMarGridding@git+ssh://git@git.noc.ac.uk/nocsurfaceprocesses/glomar\PYGZus{}gridding.git}
\end{sphinxVerbatim}


\subsection{From Source}
\label{\detokenize{getting_started:from-source}}
\sphinxAtStartPar
Alternatively, you can clone the repository and install using pip (or conda if preferred).

\begin{sphinxVerbatim}[commandchars=\\\{\}]
\PYG{g+go}{git clone git@git.noc.ac.uk/nocsurfaceprocesses/glomar\PYGZus{}gridding.git}
\PYG{g+go}{cd glomar\PYGZus{}gridding}
\PYG{g+go}{python \PYGZhy{}m venv venv}
\PYG{g+go}{source venv/bin/activate}
\PYG{g+go}{pip install \PYGZhy{}e .}
\end{sphinxVerbatim}

\sphinxstepscope


\chapter{Credits}
\label{\detokenize{authors:credits}}\label{\detokenize{authors::doc}}

\section{Development Lead}
\label{\detokenize{authors:development-lead}}\begin{itemize}
\item {} 
\sphinxAtStartPar
Agnieszka Faulkner \textless{}\sphinxhref{mailto:agfaul@noc.ac.uk}{agfaul@noc.ac.uk}\textgreater{} \sphinxhref{git.noc.ac.uk/agfaul}{@agfaul}

\item {} 
\sphinxAtStartPar
Joseph T. Siddons \textless{}\sphinxhref{mailto:josidd@noc.ac.uk}{josidd@noc.ac.uk}\textgreater{} \sphinxhref{git.noc.ac.uk/josidd}{@josidd}

\end{itemize}


\section{Contributoring Developers}
\label{\detokenize{authors:contributoring-developers}}\begin{itemize}
\item {} 
\sphinxAtStartPar
Steven Chan \textless{}\sphinxhref{mailto:stchan@noc.ac.uk}{stchan@noc.ac.uk}\textgreater{} \sphinxhref{git.noc.ac.uk/stchan}{@stchan}

\item {} 
\sphinxAtStartPar
Richard C. Cornes \textless{}\sphinxhref{mailto:rcornes@noc.ac.uk}{rcornes@noc.ac.uk}\textgreater{} \sphinxhref{git.noc.ac.uk/ricorne}{@ricorne}

\item {} 
\sphinxAtStartPar
Elizabeth C. Kent \textless{}\sphinxhref{mailto:eck@noc.ac.uk}{eck@noc.ac.uk}\textgreater{} \sphinxhref{git.noc.ac.uk/eck}{@eck}

\end{itemize}

\sphinxstepscope


\chapter{Users Guide}
\label{\detokenize{users_guide:module-glomar_gridding.grid}}\label{\detokenize{users_guide:users-guide}}\label{\detokenize{users_guide::doc}}\index{module@\spxentry{module}!glomar\_gridding.grid@\spxentry{glomar\_gridding.grid}}\index{glomar\_gridding.grid@\spxentry{glomar\_gridding.grid}!module@\spxentry{module}}

\section{Grid}
\label{\detokenize{users_guide:grid}}
\sphinxAtStartPar
Functions for creating grids and mapping observations to a grid
\index{assign\_to\_grid() (in module glomar\_gridding.grid)@\spxentry{assign\_to\_grid()}\spxextra{in module glomar\_gridding.grid}}

\begin{fulllineitems}
\phantomsection\label{\detokenize{users_guide:glomar_gridding.grid.assign_to_grid}}
\pysigstartsignatures
\pysiglinewithargsret
{\sphinxcode{\sphinxupquote{glomar\_gridding.grid.}}\sphinxbfcode{\sphinxupquote{assign\_to\_grid}}}
{\sphinxparam{\DUrole{n}{values}}\sphinxparamcomma \sphinxparam{\DUrole{n}{grid\_idx}}\sphinxparamcomma \sphinxparam{\DUrole{n}{grid}}\sphinxparamcomma \sphinxparam{\DUrole{n}{mask\_grid}\DUrole{o}{=}\DUrole{default_value}{False}}\sphinxparamcomma \sphinxparam{\DUrole{n}{mask\_value}\DUrole{o}{=}\DUrole{default_value}{nan}}}
{}
\pysigstopsignatures
\sphinxAtStartPar
Assign a vector of values to a grid, using a list of grid index values. The
default value for grid values is 0.0.

\sphinxAtStartPar
Optionally, if the grid is a mask, apply the mask to the output grid.
\begin{quote}\begin{description}
\sphinxlineitem{Parameters}\begin{itemize}
\item {} 
\sphinxAtStartPar
\sphinxstyleliteralstrong{\sphinxupquote{values}} (\sphinxstyleliteralemphasis{\sphinxupquote{pl.Series}}) \textendash{} The values to map onto the output grid.

\item {} 
\sphinxAtStartPar
\sphinxstyleliteralstrong{\sphinxupquote{grid\_idx}} (\sphinxstyleliteralemphasis{\sphinxupquote{pl.Series}}) \textendash{} The 1d index of the grid (assuming “C” style ravelling) for each value.

\item {} 
\sphinxAtStartPar
\sphinxstyleliteralstrong{\sphinxupquote{grid}} (\sphinxstyleliteralemphasis{\sphinxupquote{xarray.DataArray}}) \textendash{} The grid used to define the output grid.

\item {} 
\sphinxAtStartPar
\sphinxstyleliteralstrong{\sphinxupquote{mask\_grid}} (\sphinxstyleliteralemphasis{\sphinxupquote{bool}}) \textendash{} Optionally use values in the grid to mask the output grid.

\item {} 
\sphinxAtStartPar
\sphinxstyleliteralstrong{\sphinxupquote{mask\_value}} (\sphinxstyleliteralemphasis{\sphinxupquote{Any}}) \textendash{} The value in the grid to use for masking the output grid.

\end{itemize}

\sphinxlineitem{Returns}
\sphinxAtStartPar
\sphinxstylestrong{out\_grid} \textendash{} A new grid containing the values mapped onto the grid.

\sphinxlineitem{Return type}
\sphinxAtStartPar
xarray.DataArray

\end{description}\end{quote}

\end{fulllineitems}

\index{grid\_from\_resolution() (in module glomar\_gridding.grid)@\spxentry{grid\_from\_resolution()}\spxextra{in module glomar\_gridding.grid}}

\begin{fulllineitems}
\phantomsection\label{\detokenize{users_guide:glomar_gridding.grid.grid_from_resolution}}
\pysigstartsignatures
\pysiglinewithargsret
{\sphinxcode{\sphinxupquote{glomar\_gridding.grid.}}\sphinxbfcode{\sphinxupquote{grid\_from\_resolution}}}
{\sphinxparam{\DUrole{n}{resolution}}\sphinxparamcomma \sphinxparam{\DUrole{n}{bounds}}\sphinxparamcomma \sphinxparam{\DUrole{n}{coord\_names}}}
{}
\pysigstopsignatures
\sphinxAtStartPar
Generate a grid from a resolution value, or a list of resolutions for
given boundaries and coordinate names.

\sphinxAtStartPar
Note that all list inputs must have the same length, the ordering of values
in the lists is assumed align.
\begin{quote}\begin{description}
\sphinxlineitem{Parameters}\begin{itemize}
\item {} 
\sphinxAtStartPar
\sphinxstyleliteralstrong{\sphinxupquote{resolution}} (\sphinxstyleliteralemphasis{\sphinxupquote{float}}\sphinxstyleliteralemphasis{\sphinxupquote{ | }}\sphinxstyleliteralemphasis{\sphinxupquote{list}}\sphinxstyleliteralemphasis{\sphinxupquote{{[}}}\sphinxstyleliteralemphasis{\sphinxupquote{float}}\sphinxstyleliteralemphasis{\sphinxupquote{{]}}}) \textendash{} Resolution of the grid. Can be a single resolution value that will be
applied to all coordinates, or a list of values mapping a resolution
value to each of the coordinates.

\item {} 
\sphinxAtStartPar
\sphinxstyleliteralstrong{\sphinxupquote{bounds}} (\sphinxstyleliteralemphasis{\sphinxupquote{list}}\sphinxstyleliteralemphasis{\sphinxupquote{{[}}}\sphinxstyleliteralemphasis{\sphinxupquote{tuple}}\sphinxstyleliteralemphasis{\sphinxupquote{{[}}}\sphinxstyleliteralemphasis{\sphinxupquote{float}}\sphinxstyleliteralemphasis{\sphinxupquote{, }}\sphinxstyleliteralemphasis{\sphinxupquote{float}}\sphinxstyleliteralemphasis{\sphinxupquote{{]}}}\sphinxstyleliteralemphasis{\sphinxupquote{{]}}}) \textendash{} A list of bounds of the form \sphinxtitleref{(lower\_bound, upper\_bound)} indicating
the bounding box of the returned grid

\item {} 
\sphinxAtStartPar
\sphinxstyleliteralstrong{\sphinxupquote{coord\_names}} (\sphinxstyleliteralemphasis{\sphinxupquote{list}}\sphinxstyleliteralemphasis{\sphinxupquote{{[}}}\sphinxstyleliteralemphasis{\sphinxupquote{str}}\sphinxstyleliteralemphasis{\sphinxupquote{{]}}}) \textendash{} List of coordinate names

\end{itemize}

\sphinxlineitem{Returns}
\sphinxAtStartPar
\sphinxstylestrong{grid} \textendash{} The grid defined by the resolution and bounding box.

\sphinxlineitem{Return type}
\sphinxAtStartPar
xarray.DataArray:

\end{description}\end{quote}

\end{fulllineitems}

\index{grid\_to\_distance\_matrix() (in module glomar\_gridding.grid)@\spxentry{grid\_to\_distance\_matrix()}\spxextra{in module glomar\_gridding.grid}}

\begin{fulllineitems}
\phantomsection\label{\detokenize{users_guide:glomar_gridding.grid.grid_to_distance_matrix}}
\pysigstartsignatures
\pysiglinewithargsret
{\sphinxcode{\sphinxupquote{glomar\_gridding.grid.}}\sphinxbfcode{\sphinxupquote{grid\_to\_distance\_matrix}}}
{\sphinxparam{\DUrole{n}{grid}}\sphinxparamcomma \sphinxparam{\DUrole{n}{dist\_func=\textless{}function haversine\_distance\textgreater{}}}\sphinxparamcomma \sphinxparam{\DUrole{n}{lat\_coord=\textquotesingle{}lat\textquotesingle{}}}\sphinxparamcomma \sphinxparam{\DUrole{n}{lon\_coord=\textquotesingle{}lon\textquotesingle{}}}}
{}
\pysigstopsignatures
\sphinxAtStartPar
Calculate a distance matrix between all positions in a grid. Orientation of
latitude and longitude will be maintained in the returned distance matrix.
\begin{quote}\begin{description}
\sphinxlineitem{Parameters}\begin{itemize}
\item {} 
\sphinxAtStartPar
\sphinxstyleliteralstrong{\sphinxupquote{grid}} (\sphinxstyleliteralemphasis{\sphinxupquote{xarray.DataArray}}) \textendash{} A 2\sphinxhyphen{}d grid containing latitude and longitude indexes specified in
decimal degrees.

\item {} 
\sphinxAtStartPar
\sphinxstyleliteralstrong{\sphinxupquote{dist\_func}} (\sphinxstyleliteralemphasis{\sphinxupquote{Callable}}) \textendash{} Distance function to use to compute pairwise distances. See
glomar\_gridding.distances.calculate\_distance\_matrix for more
information.

\item {} 
\sphinxAtStartPar
\sphinxstyleliteralstrong{\sphinxupquote{lat\_coord}} (\sphinxstyleliteralemphasis{\sphinxupquote{str}}) \textendash{} Name of the latitude coordinate in the input grid.

\item {} 
\sphinxAtStartPar
\sphinxstyleliteralstrong{\sphinxupquote{lon\_coord}} (\sphinxstyleliteralemphasis{\sphinxupquote{str}}) \textendash{} Name of the longitude coordinate in the input grid.

\end{itemize}

\sphinxlineitem{Returns}
\sphinxAtStartPar
\sphinxstylestrong{dist} \textendash{} A DataArray containing the distance matrix with coordinate system
defined with grid cell index (“index\_1” and “index\_2”). The coordinates
of the original grid are also kept as coordinates related to each
index (the coordinate names are suffixed with “\_1” or “\_2” respectively.

\sphinxlineitem{Return type}
\sphinxAtStartPar
xarray.DataArray

\end{description}\end{quote}

\end{fulllineitems}

\index{map\_to\_grid() (in module glomar\_gridding.grid)@\spxentry{map\_to\_grid()}\spxextra{in module glomar\_gridding.grid}}

\begin{fulllineitems}
\phantomsection\label{\detokenize{users_guide:glomar_gridding.grid.map_to_grid}}
\pysigstartsignatures
\pysiglinewithargsret
{\sphinxcode{\sphinxupquote{glomar\_gridding.grid.}}\sphinxbfcode{\sphinxupquote{map\_to\_grid}}}
{\sphinxparam{\DUrole{n}{obs}}\sphinxparamcomma \sphinxparam{\DUrole{n}{grid}}\sphinxparamcomma \sphinxparam{\DUrole{n}{obs\_coords}\DUrole{o}{=}\DUrole{default_value}{{[}\textquotesingle{}lat\textquotesingle{}, \textquotesingle{}lon\textquotesingle{}{]}}}\sphinxparamcomma \sphinxparam{\DUrole{n}{grid\_coords}\DUrole{o}{=}\DUrole{default_value}{{[}\textquotesingle{}latitude\textquotesingle{}, \textquotesingle{}longitude\textquotesingle{}{]}}}\sphinxparamcomma \sphinxparam{\DUrole{n}{sort}\DUrole{o}{=}\DUrole{default_value}{True}}\sphinxparamcomma \sphinxparam{\DUrole{n}{bounds}\DUrole{o}{=}\DUrole{default_value}{None}}\sphinxparamcomma \sphinxparam{\DUrole{n}{add\_grid\_pts}\DUrole{o}{=}\DUrole{default_value}{True}}\sphinxparamcomma \sphinxparam{\DUrole{n}{grid\_prefix}\DUrole{o}{=}\DUrole{default_value}{\textquotesingle{}grid\_\textquotesingle{}}}}
{}
\pysigstopsignatures
\sphinxAtStartPar
Align an observation dataframe to a grid defined by an xarray DataArray.

\sphinxAtStartPar
Maps observations to the nearest grid\sphinxhyphen{}point, and sorts the data by the
1d index of the DataArray in a row\sphinxhyphen{}major format.

\sphinxAtStartPar
The grid defined by the latitude and longitude coordinates of the input
DataArray is then used as the output grid of the Gridding process.
\begin{quote}\begin{description}
\sphinxlineitem{Parameters}\begin{itemize}
\item {} 
\sphinxAtStartPar
\sphinxstyleliteralstrong{\sphinxupquote{obs}} (\sphinxstyleliteralemphasis{\sphinxupquote{polars.DataFrame}}) \textendash{} The observational DataFrame containing positional data with latitude,
longitude values within the \sphinxtitleref{obs\_latname} and \sphinxtitleref{obs\_lonname} columns
respectively. Observations are mapped to the nearest grid\sphinxhyphen{}point in the
grid.

\item {} 
\sphinxAtStartPar
\sphinxstyleliteralstrong{\sphinxupquote{grid}} (\sphinxstyleliteralemphasis{\sphinxupquote{xarray.DataArray}}) \textendash{} Contains the grid coordinates to map observations to.

\item {} 
\sphinxAtStartPar
\sphinxstyleliteralstrong{\sphinxupquote{obs\_coords}} (\sphinxstyleliteralemphasis{\sphinxupquote{list}}\sphinxstyleliteralemphasis{\sphinxupquote{{[}}}\sphinxstyleliteralemphasis{\sphinxupquote{str}}\sphinxstyleliteralemphasis{\sphinxupquote{{]}}}) \textendash{} Names of the column containing positional values in the input
observational DataFrame.

\item {} 
\sphinxAtStartPar
\sphinxstyleliteralstrong{\sphinxupquote{grid\_coords}} (\sphinxstyleliteralemphasis{\sphinxupquote{list}}\sphinxstyleliteralemphasis{\sphinxupquote{{[}}}\sphinxstyleliteralemphasis{\sphinxupquote{str}}\sphinxstyleliteralemphasis{\sphinxupquote{{]}}}) \textendash{} Names of the coordinates in the input grid DataArray used to define the
grid.

\item {} 
\sphinxAtStartPar
\sphinxstyleliteralstrong{\sphinxupquote{sort}} (\sphinxstyleliteralemphasis{\sphinxupquote{bool}}) \textendash{} Sort the observational DataFrame by the grid index

\item {} 
\sphinxAtStartPar
\sphinxstyleliteralstrong{\sphinxupquote{bounds}} (\sphinxstyleliteralemphasis{\sphinxupquote{list}}\sphinxstyleliteralemphasis{\sphinxupquote{{[}}}\sphinxstyleliteralemphasis{\sphinxupquote{tuple}}\sphinxstyleliteralemphasis{\sphinxupquote{{[}}}\sphinxstyleliteralemphasis{\sphinxupquote{float}}\sphinxstyleliteralemphasis{\sphinxupquote{, }}\sphinxstyleliteralemphasis{\sphinxupquote{float}}\sphinxstyleliteralemphasis{\sphinxupquote{{]}}}\sphinxstyleliteralemphasis{\sphinxupquote{{]} }}\sphinxstyleliteralemphasis{\sphinxupquote{| }}\sphinxstyleliteralemphasis{\sphinxupquote{None}}) \textendash{} Optionally filter the grid and DataFrame to fall within spatial bounds.
This list must have the same size and ordering as \sphinxtitleref{obs\_coords} and
\sphinxtitleref{grid\_coords} arguments.

\item {} 
\sphinxAtStartPar
\sphinxstyleliteralstrong{\sphinxupquote{add\_grid\_pts}} (\sphinxstyleliteralemphasis{\sphinxupquote{bool}}) \textendash{} Add the grid positional information to the observational DataFrame.

\item {} 
\sphinxAtStartPar
\sphinxstyleliteralstrong{\sphinxupquote{grid\_prefix}} (\sphinxstyleliteralemphasis{\sphinxupquote{str}}) \textendash{} Prefix to use for the new grid columns in the observational DataFrame.

\end{itemize}

\sphinxlineitem{Returns}
\sphinxAtStartPar
\sphinxstylestrong{obs} \textendash{} Containing additional \sphinxtitleref{grid\_*}, and \sphinxtitleref{grid\_idx} values
indicating the positions and grid index of the observation
respectively. The DataFrame is also sorted (ascendingly) by the
\sphinxtitleref{grid\_idx} columns for consistency with the gridding functions.

\sphinxlineitem{Return type}
\sphinxAtStartPar
pandas.DataFrame

\end{description}\end{quote}

\end{fulllineitems}

\index{module@\spxentry{module}!glomar\_gridding.variogram@\spxentry{glomar\_gridding.variogram}}\index{glomar\_gridding.variogram@\spxentry{glomar\_gridding.variogram}!module@\spxentry{module}}

\section{Variograms}
\label{\detokenize{users_guide:variograms}}\label{\detokenize{users_guide:module-glomar_gridding.variogram}}
\sphinxAtStartPar
Varigram classes for construction of spatial covariance structure from distance
matrices.
\index{ExponentialVariogram (class in glomar\_gridding.variogram)@\spxentry{ExponentialVariogram}\spxextra{class in glomar\_gridding.variogram}}

\begin{fulllineitems}
\phantomsection\label{\detokenize{users_guide:glomar_gridding.variogram.ExponentialVariogram}}
\pysigstartsignatures
\pysiglinewithargsret
{\sphinxbfcode{\sphinxupquote{\DUrole{k}{class}\DUrole{w}{ }}}\sphinxcode{\sphinxupquote{glomar\_gridding.variogram.}}\sphinxbfcode{\sphinxupquote{ExponentialVariogram}}}
{\sphinxparam{\DUrole{n}{psill}}\sphinxparamcomma \sphinxparam{\DUrole{n}{nugget}}\sphinxparamcomma \sphinxparam{\DUrole{n}{range}\DUrole{o}{=}\DUrole{default_value}{None}}\sphinxparamcomma \sphinxparam{\DUrole{n}{effective\_range}\DUrole{o}{=}\DUrole{default_value}{None}}}
{}
\pysigstopsignatures
\sphinxAtStartPar
Exponential Model
\begin{quote}\begin{description}
\sphinxlineitem{Parameters}\begin{itemize}
\item {} 
\sphinxAtStartPar
\sphinxstyleliteralstrong{\sphinxupquote{psill}} (\sphinxstyleliteralemphasis{\sphinxupquote{float}}\sphinxstyleliteralemphasis{\sphinxupquote{ | }}\sphinxstyleliteralemphasis{\sphinxupquote{numpy.ndarray}}) \textendash{} The variance of the variogram.

\item {} 
\sphinxAtStartPar
\sphinxstyleliteralstrong{\sphinxupquote{nugget}} (\sphinxstyleliteralemphasis{\sphinxupquote{float}}\sphinxstyleliteralemphasis{\sphinxupquote{ | }}\sphinxstyleliteralemphasis{\sphinxupquote{numpy.ndarray}})

\item {} 
\sphinxAtStartPar
\sphinxstyleliteralstrong{\sphinxupquote{effective\_range}} (\sphinxstyleliteralemphasis{\sphinxupquote{float}}\sphinxstyleliteralemphasis{\sphinxupquote{ | }}\sphinxstyleliteralemphasis{\sphinxupquote{numpy.ndarray}}\sphinxstyleliteralemphasis{\sphinxupquote{ | }}\sphinxstyleliteralemphasis{\sphinxupquote{None}})

\item {} 
\sphinxAtStartPar
\sphinxstyleliteralstrong{\sphinxupquote{range}} (\sphinxstyleliteralemphasis{\sphinxupquote{float}}\sphinxstyleliteralemphasis{\sphinxupquote{ | }}\sphinxstyleliteralemphasis{\sphinxupquote{numpy.ndarray}}\sphinxstyleliteralemphasis{\sphinxupquote{ | }}\sphinxstyleliteralemphasis{\sphinxupquote{None}})

\end{itemize}

\end{description}\end{quote}
\index{fit() (glomar\_gridding.variogram.ExponentialVariogram method)@\spxentry{fit()}\spxextra{glomar\_gridding.variogram.ExponentialVariogram method}}

\begin{fulllineitems}
\phantomsection\label{\detokenize{users_guide:glomar_gridding.variogram.ExponentialVariogram.fit}}
\pysigstartsignatures
\pysiglinewithargsret
{\sphinxbfcode{\sphinxupquote{fit}}}
{\sphinxparam{\DUrole{n}{distance\_matrix}}}
{}
\pysigstopsignatures
\sphinxAtStartPar
Fit the ExponentialVariogram model to a distance matrix
\begin{quote}\begin{description}
\sphinxlineitem{Return type}
\sphinxAtStartPar
\DUrole{sphinx_autodoc_typehints-type}{\sphinxcode{\sphinxupquote{ndarray}} | \sphinxcode{\sphinxupquote{DataArray}}}

\end{description}\end{quote}

\end{fulllineitems}


\end{fulllineitems}

\index{GaussianVariogram (class in glomar\_gridding.variogram)@\spxentry{GaussianVariogram}\spxextra{class in glomar\_gridding.variogram}}

\begin{fulllineitems}
\phantomsection\label{\detokenize{users_guide:glomar_gridding.variogram.GaussianVariogram}}
\pysigstartsignatures
\pysiglinewithargsret
{\sphinxbfcode{\sphinxupquote{\DUrole{k}{class}\DUrole{w}{ }}}\sphinxcode{\sphinxupquote{glomar\_gridding.variogram.}}\sphinxbfcode{\sphinxupquote{GaussianVariogram}}}
{\sphinxparam{\DUrole{n}{psill}}\sphinxparamcomma \sphinxparam{\DUrole{n}{nugget}}\sphinxparamcomma \sphinxparam{\DUrole{n}{effective\_range}\DUrole{o}{=}\DUrole{default_value}{None}}\sphinxparamcomma \sphinxparam{\DUrole{n}{range}\DUrole{o}{=}\DUrole{default_value}{None}}}
{}
\pysigstopsignatures
\sphinxAtStartPar
Gaussian Model
\begin{quote}\begin{description}
\sphinxlineitem{Parameters}\begin{itemize}
\item {} 
\sphinxAtStartPar
\sphinxstyleliteralstrong{\sphinxupquote{psill}} (\sphinxstyleliteralemphasis{\sphinxupquote{float}}\sphinxstyleliteralemphasis{\sphinxupquote{ | }}\sphinxstyleliteralemphasis{\sphinxupquote{np.ndarray}}) \textendash{} The variance of the variogram.

\item {} 
\sphinxAtStartPar
\sphinxstyleliteralstrong{\sphinxupquote{nugget}} (\sphinxstyleliteralemphasis{\sphinxupquote{float}}\sphinxstyleliteralemphasis{\sphinxupquote{ | }}\sphinxstyleliteralemphasis{\sphinxupquote{np.ndarray}})

\item {} 
\sphinxAtStartPar
\sphinxstyleliteralstrong{\sphinxupquote{effective\_range}} (\sphinxstyleliteralemphasis{\sphinxupquote{float}}\sphinxstyleliteralemphasis{\sphinxupquote{ | }}\sphinxstyleliteralemphasis{\sphinxupquote{np.ndarray}}\sphinxstyleliteralemphasis{\sphinxupquote{ | }}\sphinxstyleliteralemphasis{\sphinxupquote{None}})

\item {} 
\sphinxAtStartPar
\sphinxstyleliteralstrong{\sphinxupquote{range}} (\sphinxstyleliteralemphasis{\sphinxupquote{float}}\sphinxstyleliteralemphasis{\sphinxupquote{ | }}\sphinxstyleliteralemphasis{\sphinxupquote{np.ndarray}}\sphinxstyleliteralemphasis{\sphinxupquote{ | }}\sphinxstyleliteralemphasis{\sphinxupquote{None}})

\end{itemize}

\end{description}\end{quote}
\index{fit() (glomar\_gridding.variogram.GaussianVariogram method)@\spxentry{fit()}\spxextra{glomar\_gridding.variogram.GaussianVariogram method}}

\begin{fulllineitems}
\phantomsection\label{\detokenize{users_guide:glomar_gridding.variogram.GaussianVariogram.fit}}
\pysigstartsignatures
\pysiglinewithargsret
{\sphinxbfcode{\sphinxupquote{fit}}}
{\sphinxparam{\DUrole{n}{distance\_matrix}}}
{}
\pysigstopsignatures
\sphinxAtStartPar
Fit the GaussianVariogram model to a distance matrix
\begin{quote}\begin{description}
\sphinxlineitem{Return type}
\sphinxAtStartPar
\DUrole{sphinx_autodoc_typehints-type}{\sphinxcode{\sphinxupquote{ndarray}} | \sphinxcode{\sphinxupquote{DataArray}}}

\end{description}\end{quote}

\end{fulllineitems}


\end{fulllineitems}

\index{LinearVariogram (class in glomar\_gridding.variogram)@\spxentry{LinearVariogram}\spxextra{class in glomar\_gridding.variogram}}

\begin{fulllineitems}
\phantomsection\label{\detokenize{users_guide:glomar_gridding.variogram.LinearVariogram}}
\pysigstartsignatures
\pysiglinewithargsret
{\sphinxbfcode{\sphinxupquote{\DUrole{k}{class}\DUrole{w}{ }}}\sphinxcode{\sphinxupquote{glomar\_gridding.variogram.}}\sphinxbfcode{\sphinxupquote{LinearVariogram}}}
{\sphinxparam{\DUrole{n}{slope}}\sphinxparamcomma \sphinxparam{\DUrole{n}{nugget}}}
{}
\pysigstopsignatures
\sphinxAtStartPar
Linear model
\begin{quote}\begin{description}
\sphinxlineitem{Parameters}\begin{itemize}
\item {} 
\sphinxAtStartPar
\sphinxstyleliteralstrong{\sphinxupquote{slope}} (\sphinxstyleliteralemphasis{\sphinxupquote{float}}\sphinxstyleliteralemphasis{\sphinxupquote{ | }}\sphinxstyleliteralemphasis{\sphinxupquote{np.ndarray}})

\item {} 
\sphinxAtStartPar
\sphinxstyleliteralstrong{\sphinxupquote{nugget}} (\sphinxstyleliteralemphasis{\sphinxupquote{float}}\sphinxstyleliteralemphasis{\sphinxupquote{ | }}\sphinxstyleliteralemphasis{\sphinxupquote{np.ndarray}})

\end{itemize}

\end{description}\end{quote}
\index{fit() (glomar\_gridding.variogram.LinearVariogram method)@\spxentry{fit()}\spxextra{glomar\_gridding.variogram.LinearVariogram method}}

\begin{fulllineitems}
\phantomsection\label{\detokenize{users_guide:glomar_gridding.variogram.LinearVariogram.fit}}
\pysigstartsignatures
\pysiglinewithargsret
{\sphinxbfcode{\sphinxupquote{fit}}}
{\sphinxparam{\DUrole{n}{distance\_matrix}}}
{}
\pysigstopsignatures
\sphinxAtStartPar
Fit the LinearVariogram model to a distance matrix
\begin{quote}\begin{description}
\sphinxlineitem{Return type}
\sphinxAtStartPar
\DUrole{sphinx_autodoc_typehints-type}{\sphinxcode{\sphinxupquote{ndarray}} | \sphinxcode{\sphinxupquote{DataArray}}}

\end{description}\end{quote}

\end{fulllineitems}


\end{fulllineitems}

\index{MaternVariogram (class in glomar\_gridding.variogram)@\spxentry{MaternVariogram}\spxextra{class in glomar\_gridding.variogram}}

\begin{fulllineitems}
\phantomsection\label{\detokenize{users_guide:glomar_gridding.variogram.MaternVariogram}}
\pysigstartsignatures
\pysiglinewithargsret
{\sphinxbfcode{\sphinxupquote{\DUrole{k}{class}\DUrole{w}{ }}}\sphinxcode{\sphinxupquote{glomar\_gridding.variogram.}}\sphinxbfcode{\sphinxupquote{MaternVariogram}}}
{\sphinxparam{\DUrole{n}{psill}}\sphinxparamcomma \sphinxparam{\DUrole{n}{nugget}}\sphinxparamcomma \sphinxparam{\DUrole{n}{effective\_range}\DUrole{o}{=}\DUrole{default_value}{None}}\sphinxparamcomma \sphinxparam{\DUrole{n}{range}\DUrole{o}{=}\DUrole{default_value}{None}}\sphinxparamcomma \sphinxparam{\DUrole{n}{nu}\DUrole{o}{=}\DUrole{default_value}{0.5}}\sphinxparamcomma \sphinxparam{\DUrole{n}{method}\DUrole{o}{=}\DUrole{default_value}{\textquotesingle{}sklearn\textquotesingle{}}}}
{}
\pysigstopsignatures
\sphinxAtStartPar
Matern Models

\sphinxAtStartPar
Same args as the Variogram classes with additional nu, method parameters.

\sphinxAtStartPar
Sklearn:
\begin{enumerate}
\sphinxsetlistlabels{\arabic}{enumi}{enumii}{}{)}%
\item {} 
\sphinxAtStartPar
This is called “sklearn” because if d/range = 1.0 and nu=0.5, it gives
1/e correlation…

\item {} 
\sphinxAtStartPar
This is NOT the same formulation as in GSTAT nor in papers about
non\sphinxhyphen{}stationary anistropic covariance models (aka Karspeck paper).

\item {} 
\sphinxAtStartPar
It is perhaps the most intitutive (because of (1)) and is used in sklearn
GP and HadCRUT5 and other UKMO dataset.

\item {} 
\sphinxAtStartPar
nu defaults to 0.5 (exponential; used in HADSST4 and our kriging).
HadCRUT5 uses 1.5.

\item {} 
\sphinxAtStartPar
The “2” is inside the square root for middle and right.

\end{enumerate}

\sphinxAtStartPar
Reference; see chapter 4.2 of:
Rasmussen, C. E., \& Williams, C. K. I. (2005).
Gaussian Processes for Machine Learning. The MIT Press.
\sphinxurl{https://doi.org/10.7551/mitpress/3206.001.0001}

\sphinxAtStartPar
GeoStatic:

\sphinxAtStartPar
Similar to Sklearn MaternVariogram model but uses the range scaling in
gstat.
Note: there are no square root 2 or nu in middle and right

\sphinxAtStartPar
Yields the same answer to sklearn MaternVariogram if nu==0.5
but are otherwise different.

\sphinxAtStartPar
Karspeck:

\sphinxAtStartPar
Similar to Sklearn MaternVariogram model but uses the form in Karspeck paper
Note: Note the 2 is outside the square root for middle and right
e\sphinxhyphen{}folding distance is now at d/SQRT(2) for nu=0.5
\begin{quote}\begin{description}
\sphinxlineitem{Parameters}\begin{itemize}
\item {} 
\sphinxAtStartPar
\sphinxstyleliteralstrong{\sphinxupquote{psill}} (\sphinxstyleliteralemphasis{\sphinxupquote{float}}\sphinxstyleliteralemphasis{\sphinxupquote{ | }}\sphinxstyleliteralemphasis{\sphinxupquote{np.ndarray}}) \textendash{} Sill of the variogram where it will flatten out. Values in the variogram
will not exceed psill + nugget. This value is the variance.

\item {} 
\sphinxAtStartPar
\sphinxstyleliteralstrong{\sphinxupquote{nugget}} (\sphinxstyleliteralemphasis{\sphinxupquote{float}}\sphinxstyleliteralemphasis{\sphinxupquote{ | }}\sphinxstyleliteralemphasis{\sphinxupquote{np.ndarray}}) \textendash{} The value of the independent variable at distance 0

\item {} 
\sphinxAtStartPar
\sphinxstyleliteralstrong{\sphinxupquote{effective\_range}} (\sphinxstyleliteralemphasis{\sphinxupquote{float}}\sphinxstyleliteralemphasis{\sphinxupquote{ | }}\sphinxstyleliteralemphasis{\sphinxupquote{np.ndarray}}\sphinxstyleliteralemphasis{\sphinxupquote{ | }}\sphinxstyleliteralemphasis{\sphinxupquote{None}}) \textendash{} Effective Range, this is the lag where 95\% of ths sill are exceeded.
This is not the range parameter, which is defined as r/3 if nu \textless{} 0.5 or
nu \textgreater{} 10, otherwise r/2 (where r is the effective range). One of
effective\_range and range must be set.

\item {} 
\sphinxAtStartPar
\sphinxstyleliteralstrong{\sphinxupquote{range}} (\sphinxstyleliteralemphasis{\sphinxupquote{float}}\sphinxstyleliteralemphasis{\sphinxupquote{ | }}\sphinxstyleliteralemphasis{\sphinxupquote{ndarray}}\sphinxstyleliteralemphasis{\sphinxupquote{ | }}\sphinxstyleliteralemphasis{\sphinxupquote{None}}) \textendash{} The range parameter. One of range and effective\_range must be set. If
range is not set, it will be computed from effective\_range.

\item {} 
\sphinxAtStartPar
\sphinxstyleliteralstrong{\sphinxupquote{nu}} (\sphinxstyleliteralemphasis{\sphinxupquote{float}}\sphinxstyleliteralemphasis{\sphinxupquote{ | }}\sphinxstyleliteralemphasis{\sphinxupquote{np.ndarray}}) \textendash{} Smoothing parameter, shapes to a smooth or rough variogram function

\item {} 
\sphinxAtStartPar
\sphinxstyleliteralstrong{\sphinxupquote{method}} (\sphinxstyleliteralemphasis{\sphinxupquote{MaternModel}}) \textendash{} One of “sklearn”, “gstat”, or “karspeck”
sklearn: \sphinxurl{https://scikit-learn.org/stable/modules/generated/sklearn.gaussian\_process.kernels.Matern.html\#sklearn.gaussian\_process.kernels.Matern}
gstat: \sphinxurl{https://scikit-gstat.readthedocs.io/en/latest/reference/models.html\#matern-model}
karspeck: \sphinxurl{https://rmets.onlinelibrary.wiley.com/doi/10.1002/qj.900}

\end{itemize}

\end{description}\end{quote}
\index{fit() (glomar\_gridding.variogram.MaternVariogram method)@\spxentry{fit()}\spxextra{glomar\_gridding.variogram.MaternVariogram method}}

\begin{fulllineitems}
\phantomsection\label{\detokenize{users_guide:glomar_gridding.variogram.MaternVariogram.fit}}
\pysigstartsignatures
\pysiglinewithargsret
{\sphinxbfcode{\sphinxupquote{fit}}}
{\sphinxparam{\DUrole{n}{distance\_matrix}}}
{}
\pysigstopsignatures
\sphinxAtStartPar
Fit the MaternVariogram model to a distance matrix
\begin{quote}\begin{description}
\sphinxlineitem{Return type}
\sphinxAtStartPar
\DUrole{sphinx_autodoc_typehints-type}{\sphinxcode{\sphinxupquote{ndarray}} | \sphinxcode{\sphinxupquote{DataArray}}}

\end{description}\end{quote}

\end{fulllineitems}


\end{fulllineitems}

\index{PowerVariogram (class in glomar\_gridding.variogram)@\spxentry{PowerVariogram}\spxextra{class in glomar\_gridding.variogram}}

\begin{fulllineitems}
\phantomsection\label{\detokenize{users_guide:glomar_gridding.variogram.PowerVariogram}}
\pysigstartsignatures
\pysiglinewithargsret
{\sphinxbfcode{\sphinxupquote{\DUrole{k}{class}\DUrole{w}{ }}}\sphinxcode{\sphinxupquote{glomar\_gridding.variogram.}}\sphinxbfcode{\sphinxupquote{PowerVariogram}}}
{\sphinxparam{\DUrole{n}{scale}}\sphinxparamcomma \sphinxparam{\DUrole{n}{exponent}}\sphinxparamcomma \sphinxparam{\DUrole{n}{nugget}}}
{}
\pysigstopsignatures
\sphinxAtStartPar
Power model
\begin{quote}\begin{description}
\sphinxlineitem{Parameters}\begin{itemize}
\item {} 
\sphinxAtStartPar
\sphinxstyleliteralstrong{\sphinxupquote{scale}} (\sphinxstyleliteralemphasis{\sphinxupquote{float}}\sphinxstyleliteralemphasis{\sphinxupquote{ | }}\sphinxstyleliteralemphasis{\sphinxupquote{np.ndarray}})

\item {} 
\sphinxAtStartPar
\sphinxstyleliteralstrong{\sphinxupquote{exponent}} (\sphinxstyleliteralemphasis{\sphinxupquote{float}}\sphinxstyleliteralemphasis{\sphinxupquote{ | }}\sphinxstyleliteralemphasis{\sphinxupquote{np.ndarray}})

\item {} 
\sphinxAtStartPar
\sphinxstyleliteralstrong{\sphinxupquote{nugget}} (\sphinxstyleliteralemphasis{\sphinxupquote{float}}\sphinxstyleliteralemphasis{\sphinxupquote{ | }}\sphinxstyleliteralemphasis{\sphinxupquote{np.ndarray}})

\end{itemize}

\end{description}\end{quote}
\index{fit() (glomar\_gridding.variogram.PowerVariogram method)@\spxentry{fit()}\spxextra{glomar\_gridding.variogram.PowerVariogram method}}

\begin{fulllineitems}
\phantomsection\label{\detokenize{users_guide:glomar_gridding.variogram.PowerVariogram.fit}}
\pysigstartsignatures
\pysiglinewithargsret
{\sphinxbfcode{\sphinxupquote{fit}}}
{\sphinxparam{\DUrole{n}{distance\_matrix}}}
{}
\pysigstopsignatures
\sphinxAtStartPar
Fit the PowerVariogram model to a distance matrix
\begin{quote}\begin{description}
\sphinxlineitem{Return type}
\sphinxAtStartPar
\DUrole{sphinx_autodoc_typehints-type}{\sphinxcode{\sphinxupquote{ndarray}} | \sphinxcode{\sphinxupquote{DataArray}}}

\end{description}\end{quote}

\end{fulllineitems}


\end{fulllineitems}

\index{Variogram (class in glomar\_gridding.variogram)@\spxentry{Variogram}\spxextra{class in glomar\_gridding.variogram}}

\begin{fulllineitems}
\phantomsection\label{\detokenize{users_guide:glomar_gridding.variogram.Variogram}}
\pysigstartsignatures
\pysigline
{\sphinxbfcode{\sphinxupquote{\DUrole{k}{class}\DUrole{w}{ }}}\sphinxcode{\sphinxupquote{glomar\_gridding.variogram.}}\sphinxbfcode{\sphinxupquote{Variogram}}}
\pysigstopsignatures
\sphinxAtStartPar
Place holder
\index{fit() (glomar\_gridding.variogram.Variogram method)@\spxentry{fit()}\spxextra{glomar\_gridding.variogram.Variogram method}}

\begin{fulllineitems}
\phantomsection\label{\detokenize{users_guide:glomar_gridding.variogram.Variogram.fit}}
\pysigstartsignatures
\pysiglinewithargsret
{\sphinxbfcode{\sphinxupquote{fit}}}
{\sphinxparam{\DUrole{n}{distance\_matrix}}}
{}
\pysigstopsignatures
\sphinxAtStartPar
Fit the Variogram model to a distance matrix
\begin{quote}\begin{description}
\sphinxlineitem{Return type}
\sphinxAtStartPar
\DUrole{sphinx_autodoc_typehints-type}{\sphinxcode{\sphinxupquote{ndarray}} | \sphinxcode{\sphinxupquote{DataArray}}}

\end{description}\end{quote}

\end{fulllineitems}


\end{fulllineitems}

\index{variogram\_to\_covariance() (in module glomar\_gridding.variogram)@\spxentry{variogram\_to\_covariance()}\spxextra{in module glomar\_gridding.variogram}}

\begin{fulllineitems}
\phantomsection\label{\detokenize{users_guide:glomar_gridding.variogram.variogram_to_covariance}}
\pysigstartsignatures
\pysiglinewithargsret
{\sphinxcode{\sphinxupquote{glomar\_gridding.variogram.}}\sphinxbfcode{\sphinxupquote{variogram\_to\_covariance}}}
{\sphinxparam{\DUrole{n}{variogram}}\sphinxparamcomma \sphinxparam{\DUrole{n}{variance}}}
{}
\pysigstopsignatures
\sphinxAtStartPar
Convert a variogram matrix to a covariance matrix.
\begin{description}
\sphinxlineitem{This is given by:}
\sphinxAtStartPar
covariance = variance \sphinxhyphen{} variogram

\end{description}
\begin{quote}\begin{description}
\sphinxlineitem{Parameters}\begin{itemize}
\item {} 
\sphinxAtStartPar
\sphinxstyleliteralstrong{\sphinxupquote{variogram}} (\sphinxstyleliteralemphasis{\sphinxupquote{numpy.ndarray}}\sphinxstyleliteralemphasis{\sphinxupquote{ | }}\sphinxstyleliteralemphasis{\sphinxupquote{xarray.DataArray}}) \textendash{} The variogram matrix, output of Variogram.fit.

\item {} 
\sphinxAtStartPar
\sphinxstyleliteralstrong{\sphinxupquote{variance}} (\sphinxstyleliteralemphasis{\sphinxupquote{numpy.ndarray}}\sphinxstyleliteralemphasis{\sphinxupquote{ | }}\sphinxstyleliteralemphasis{\sphinxupquote{float}}) \textendash{} The variance

\end{itemize}

\sphinxlineitem{Returns}
\sphinxAtStartPar
\sphinxstylestrong{cov} \textendash{} The covariance matrix

\sphinxlineitem{Return type}
\sphinxAtStartPar
numpy.ndarray | xarray.DataArray

\end{description}\end{quote}

\end{fulllineitems}

\index{module@\spxentry{module}!glomar\_gridding.kriging@\spxentry{glomar\_gridding.kriging}}\index{glomar\_gridding.kriging@\spxentry{glomar\_gridding.kriging}!module@\spxentry{module}}

\section{Kriging}
\label{\detokenize{users_guide:kriging}}\label{\detokenize{users_guide:module-glomar_gridding.kriging}}
\sphinxAtStartPar
Functions for performing Kriging.

\sphinxAtStartPar
Interpolation using a Gaussian Process. Available methods are Simple and
Ordinary Kriging.
\index{get\_spatial\_mean() (in module glomar\_gridding.kriging)@\spxentry{get\_spatial\_mean()}\spxextra{in module glomar\_gridding.kriging}}

\begin{fulllineitems}
\phantomsection\label{\detokenize{users_guide:glomar_gridding.kriging.get_spatial_mean}}
\pysigstartsignatures
\pysiglinewithargsret
{\sphinxcode{\sphinxupquote{glomar\_gridding.kriging.}}\sphinxbfcode{\sphinxupquote{get\_spatial\_mean}}}
{\sphinxparam{\DUrole{n}{grid\_obs}}\sphinxparamcomma \sphinxparam{\DUrole{n}{covx}}}
{}
\pysigstopsignatures
\sphinxAtStartPar
Compute the spatial mean accounting for auto\sphinxhyphen{}correlation.
\begin{quote}\begin{description}
\sphinxlineitem{Parameters}\begin{itemize}
\item {} 
\sphinxAtStartPar
\sphinxstyleliteralstrong{\sphinxupquote{grid\_obs}} (\sphinxstyleliteralemphasis{\sphinxupquote{np.ndarray}}) \textendash{} Vector containing observations

\item {} 
\sphinxAtStartPar
\sphinxstyleliteralstrong{\sphinxupquote{covx}} (\sphinxstyleliteralemphasis{\sphinxupquote{np.ndarray}}) \textendash{} Observation covariance matrix

\end{itemize}

\sphinxlineitem{Return type}
\sphinxAtStartPar
\DUrole{sphinx_autodoc_typehints-type}{\sphinxcode{\sphinxupquote{float}}}

\sphinxlineitem{Returns}
\sphinxAtStartPar
\begin{itemize}
\item {} 
\sphinxAtStartPar
\sphinxstylestrong{spatial\_mean} (\sphinxstyleemphasis{float}) \textendash{} The spatial mean defined as (1\textasciicircum{}T x C\textasciicircum{}\{\sphinxhyphen{}1\} x 1)\textasciicircum{}\{\sphinxhyphen{}1\} * (1\textasciicircum{}T x C\textasciicircum{}\{\sphinxhyphen{}1\} x z)

\item {} 
\sphinxAtStartPar
\sphinxstyleemphasis{Reference}

\item {} 
\sphinxAtStartPar
\sphinxstyleemphasis{———}

\item {} 
\sphinxAtStartPar
\sphinxstylestrong{https} (\sphinxstyleemphasis{//www.css.cornell.edu/faculty/dgr2/\_static/files/distance\_ed\_geostats/ov5.pdf})

\end{itemize}


\end{description}\end{quote}

\end{fulllineitems}

\index{get\_unmasked\_obs\_indices() (in module glomar\_gridding.kriging)@\spxentry{get\_unmasked\_obs\_indices()}\spxextra{in module glomar\_gridding.kriging}}

\begin{fulllineitems}
\phantomsection\label{\detokenize{users_guide:glomar_gridding.kriging.get_unmasked_obs_indices}}
\pysigstartsignatures
\pysiglinewithargsret
{\sphinxcode{\sphinxupquote{glomar\_gridding.kriging.}}\sphinxbfcode{\sphinxupquote{get\_unmasked\_obs\_indices}}}
{\sphinxparam{\DUrole{n}{unmask\_idx}}\sphinxparamcomma \sphinxparam{\DUrole{n}{unique\_obs\_idx}}}
{}
\pysigstopsignatures
\sphinxAtStartPar
Get grid indices with observations from un\sphinxhyphen{}masked grid\sphinxhyphen{}box indices and
unique grid\sphinxhyphen{}box indices with observations.
\begin{quote}\begin{description}
\sphinxlineitem{Parameters}\begin{itemize}
\item {} 
\sphinxAtStartPar
\sphinxstyleliteralstrong{\sphinxupquote{unmask\_idx}} (\sphinxstyleliteralemphasis{\sphinxupquote{np.ndarray}}\sphinxstyleliteralemphasis{\sphinxupquote{{[}}}\sphinxstyleliteralemphasis{\sphinxupquote{int}}\sphinxstyleliteralemphasis{\sphinxupquote{{]}}}) \textendash{} List of all unmasked grid\sphinxhyphen{}box indices.

\item {} 
\sphinxAtStartPar
\sphinxstyleliteralstrong{\sphinxupquote{unique\_obs\_idx}} (\sphinxstyleliteralemphasis{\sphinxupquote{np.ndarray}}\sphinxstyleliteralemphasis{\sphinxupquote{{[}}}\sphinxstyleliteralemphasis{\sphinxupquote{int}}\sphinxstyleliteralemphasis{\sphinxupquote{{]}}}) \textendash{} Indices of grid\sphinxhyphen{}boxes with observations.

\end{itemize}

\sphinxlineitem{Returns}
\sphinxAtStartPar
\sphinxstylestrong{obs\_idx} \textendash{} Subset of grid\sphinxhyphen{}box indices containing observations that are unmasked.

\sphinxlineitem{Return type}
\sphinxAtStartPar
np.ndarray{[}int{]}

\end{description}\end{quote}

\end{fulllineitems}

\index{kriging() (in module glomar\_gridding.kriging)@\spxentry{kriging()}\spxextra{in module glomar\_gridding.kriging}}

\begin{fulllineitems}
\phantomsection\label{\detokenize{users_guide:glomar_gridding.kriging.kriging}}
\pysigstartsignatures
\pysiglinewithargsret
{\sphinxcode{\sphinxupquote{glomar\_gridding.kriging.}}\sphinxbfcode{\sphinxupquote{kriging}}}
{\sphinxparam{\DUrole{n}{obs\_idx}}\sphinxparamcomma \sphinxparam{\DUrole{n}{weights}}\sphinxparamcomma \sphinxparam{\DUrole{n}{obs}}\sphinxparamcomma \sphinxparam{\DUrole{n}{interp\_cov}}\sphinxparamcomma \sphinxparam{\DUrole{n}{error\_cov}}\sphinxparamcomma \sphinxparam{\DUrole{n}{remove\_obs\_mean}\DUrole{o}{=}\DUrole{default_value}{0}}\sphinxparamcomma \sphinxparam{\DUrole{n}{obs\_bias}\DUrole{o}{=}\DUrole{default_value}{None}}\sphinxparamcomma \sphinxparam{\DUrole{n}{method}\DUrole{o}{=}\DUrole{default_value}{\textquotesingle{}simple\textquotesingle{}}}}
{}
\pysigstopsignatures
\sphinxAtStartPar
Perform Kriging using a chosen method.

\sphinxAtStartPar
Get array of krigged observations and anomalies for all grid points in the
domain.
\begin{quote}\begin{description}
\sphinxlineitem{Parameters}\begin{itemize}
\item {} 
\sphinxAtStartPar
\sphinxstyleliteralstrong{\sphinxupquote{obs\_idx}} (\sphinxstyleliteralemphasis{\sphinxupquote{np.ndarray}}\sphinxstyleliteralemphasis{\sphinxupquote{{[}}}\sphinxstyleliteralemphasis{\sphinxupquote{int}}\sphinxstyleliteralemphasis{\sphinxupquote{{]}}}) \textendash{} Grid indices with observations. It is expected that this should be an
ordering that lines up with the 1st dimension of weights. If
\sphinxtitleref{observations.dist\_weights} or \sphinxtitleref{observations.get\_weights} was used to
get the weights then this is the ordering of
\sphinxtitleref{sorted(df{[}“gridbox”{]}.unique())}, which is a sorting on lat and lon

\item {} 
\sphinxAtStartPar
\sphinxstyleliteralstrong{\sphinxupquote{weights}} (\sphinxstyleliteralemphasis{\sphinxupquote{np.ndarray}}\sphinxstyleliteralemphasis{\sphinxupquote{{[}}}\sphinxstyleliteralemphasis{\sphinxupquote{float}}\sphinxstyleliteralemphasis{\sphinxupquote{{]}}}) \textendash{} Weight matrix (inverse of counts of observations).

\item {} 
\sphinxAtStartPar
\sphinxstyleliteralstrong{\sphinxupquote{obs}} (\sphinxstyleliteralemphasis{\sphinxupquote{np.ndarray}}\sphinxstyleliteralemphasis{\sphinxupquote{{[}}}\sphinxstyleliteralemphasis{\sphinxupquote{float}}\sphinxstyleliteralemphasis{\sphinxupquote{{]}}}) \textendash{} All point observations/measurements for the chosen date.

\item {} 
\sphinxAtStartPar
\sphinxstyleliteralstrong{\sphinxupquote{interp\_cov}} (\sphinxstyleliteralemphasis{\sphinxupquote{np.ndarray}}\sphinxstyleliteralemphasis{\sphinxupquote{{[}}}\sphinxstyleliteralemphasis{\sphinxupquote{float}}\sphinxstyleliteralemphasis{\sphinxupquote{{]}}}) \textendash{} interpolation covariance of all output grid points (each point in time
and all points against each other).

\item {} 
\sphinxAtStartPar
\sphinxstyleliteralstrong{\sphinxupquote{error\_cov}} (\sphinxstyleliteralemphasis{\sphinxupquote{np.ndarray}}\sphinxstyleliteralemphasis{\sphinxupquote{{[}}}\sphinxstyleliteralemphasis{\sphinxupquote{float}}\sphinxstyleliteralemphasis{\sphinxupquote{{]}}}) \textendash{} Measurement/Error covariance matrix.

\item {} 
\sphinxAtStartPar
\sphinxstyleliteralstrong{\sphinxupquote{remove\_obs\_mean}} (\sphinxstyleliteralemphasis{\sphinxupquote{int}}) \textendash{} Should the mean or median from grib\_obs be removed and added back onto
grib\_obs?
0 = No (default action)
1 = the mean is removed
2 = the median is removed
3 = the spatial meam os removed

\item {} 
\sphinxAtStartPar
\sphinxstyleliteralstrong{\sphinxupquote{obs\_bias}} (\sphinxstyleliteralemphasis{\sphinxupquote{np.ndarray}}\sphinxstyleliteralemphasis{\sphinxupquote{{[}}}\sphinxstyleliteralemphasis{\sphinxupquote{float}}\sphinxstyleliteralemphasis{\sphinxupquote{{]} }}\sphinxstyleliteralemphasis{\sphinxupquote{| }}\sphinxstyleliteralemphasis{\sphinxupquote{None}}) \textendash{} Bias of all measurement points for a chosen date (corresponds to x\_obs).

\item {} 
\sphinxAtStartPar
\sphinxstyleliteralstrong{\sphinxupquote{method}} (\sphinxstyleliteralemphasis{\sphinxupquote{KrigMethod}}) \textendash{} The kriging method to use to fill in the output grid. One of “simple”
or “ordinary”.

\end{itemize}

\sphinxlineitem{Return type}
\sphinxAtStartPar
\DUrole{sphinx_autodoc_typehints-type}{\sphinxcode{\sphinxupquote{tuple}}{[}\sphinxcode{\sphinxupquote{ndarray}}, \sphinxcode{\sphinxupquote{ndarray}}{]}}

\sphinxlineitem{Returns}
\sphinxAtStartPar
\begin{itemize}
\item {} 
\sphinxAtStartPar
\sphinxstylestrong{z\_obs} (\sphinxstyleemphasis{np.ndarray{[}float{]}}) \textendash{} Full set of values for the whole domain derived from the observation
points using the chosen kriging method.

\item {} 
\sphinxAtStartPar
\sphinxstylestrong{dz} (\sphinxstyleemphasis{np.ndarray{[}float{]}}) \textendash{} Uncertainty associated with the chosen kriging method.

\end{itemize}


\end{description}\end{quote}

\end{fulllineitems}

\index{kriging\_ordinary() (in module glomar\_gridding.kriging)@\spxentry{kriging\_ordinary()}\spxextra{in module glomar\_gridding.kriging}}

\begin{fulllineitems}
\phantomsection\label{\detokenize{users_guide:glomar_gridding.kriging.kriging_ordinary}}
\pysigstartsignatures
\pysiglinewithargsret
{\sphinxcode{\sphinxupquote{glomar\_gridding.kriging.}}\sphinxbfcode{\sphinxupquote{kriging\_ordinary}}}
{\sphinxparam{\DUrole{n}{S}}\sphinxparamcomma \sphinxparam{\DUrole{n}{Ss}}\sphinxparamcomma \sphinxparam{\DUrole{n}{grid\_obs}}\sphinxparamcomma \sphinxparam{\DUrole{n}{interp\_cov}}}
{}
\pysigstopsignatures
\sphinxAtStartPar
Perform Ordinary Kriging with unknown but constant mean.
\begin{quote}\begin{description}
\sphinxlineitem{Parameters}\begin{itemize}
\item {} 
\sphinxAtStartPar
\sphinxstyleliteralstrong{\sphinxupquote{S}} (\sphinxstyleliteralemphasis{\sphinxupquote{np.ndarray}}\sphinxstyleliteralemphasis{\sphinxupquote{{[}}}\sphinxstyleliteralemphasis{\sphinxupquote{float}}\sphinxstyleliteralemphasis{\sphinxupquote{{]}}}) \textendash{} Spatial covariance between all measured grid points plus the
covariance due to measurements (i.e. measurement noise, bias noise, and
sampling noise).

\item {} 
\sphinxAtStartPar
\sphinxstyleliteralstrong{\sphinxupquote{Ss}} (\sphinxstyleliteralemphasis{\sphinxupquote{np.ndarray}}\sphinxstyleliteralemphasis{\sphinxupquote{{[}}}\sphinxstyleliteralemphasis{\sphinxupquote{float}}\sphinxstyleliteralemphasis{\sphinxupquote{{]}}}) \textendash{} Covariance between the all (predicted) grid points and measured points.

\item {} 
\sphinxAtStartPar
\sphinxstyleliteralstrong{\sphinxupquote{grid\_obs}} (\sphinxstyleliteralemphasis{\sphinxupquote{np.ndarray}}\sphinxstyleliteralemphasis{\sphinxupquote{{[}}}\sphinxstyleliteralemphasis{\sphinxupquote{float}}\sphinxstyleliteralemphasis{\sphinxupquote{{]}}}) \textendash{} Gridded measurements (all measurement points averaged onto the output
gridboxes).

\item {} 
\sphinxAtStartPar
\sphinxstyleliteralstrong{\sphinxupquote{interp\_cov}} (\sphinxstyleliteralemphasis{\sphinxupquote{np.ndarray}}\sphinxstyleliteralemphasis{\sphinxupquote{{[}}}\sphinxstyleliteralemphasis{\sphinxupquote{float}}\sphinxstyleliteralemphasis{\sphinxupquote{{]}}}) \textendash{} Interpolation covariance of all output grid points (each point in time
and all points against each other).

\end{itemize}

\sphinxlineitem{Return type}
\sphinxAtStartPar
\DUrole{sphinx_autodoc_typehints-type}{\sphinxcode{\sphinxupquote{tuple}}{[}\sphinxcode{\sphinxupquote{ndarray}}, \sphinxcode{\sphinxupquote{ndarray}}{]}}

\sphinxlineitem{Returns}
\sphinxAtStartPar
\begin{itemize}
\item {} 
\sphinxAtStartPar
\sphinxstylestrong{z\_obs} (\sphinxstyleemphasis{np.ndarray{[}float{]}}) \textendash{} Full set of values for the whole domain derived from the observation
points using ordinary kriging.

\item {} 
\sphinxAtStartPar
\sphinxstylestrong{dz} (\sphinxstyleemphasis{np.ndarray{[}float{]}}) \textendash{} Uncertainty associated with the ordinary kriging.

\end{itemize}


\end{description}\end{quote}

\end{fulllineitems}

\index{kriging\_simple() (in module glomar\_gridding.kriging)@\spxentry{kriging\_simple()}\spxextra{in module glomar\_gridding.kriging}}

\begin{fulllineitems}
\phantomsection\label{\detokenize{users_guide:glomar_gridding.kriging.kriging_simple}}
\pysigstartsignatures
\pysiglinewithargsret
{\sphinxcode{\sphinxupquote{glomar\_gridding.kriging.}}\sphinxbfcode{\sphinxupquote{kriging\_simple}}}
{\sphinxparam{\DUrole{n}{S}}\sphinxparamcomma \sphinxparam{\DUrole{n}{Ss}}\sphinxparamcomma \sphinxparam{\DUrole{n}{grid\_obs}}\sphinxparamcomma \sphinxparam{\DUrole{n}{interp\_cov}}\sphinxparamcomma \sphinxparam{\DUrole{n}{mean}\DUrole{o}{=}\DUrole{default_value}{0.0}}}
{}
\pysigstopsignatures
\sphinxAtStartPar
Perform Simple Kriging assuming a constant known mean.
\begin{quote}\begin{description}
\sphinxlineitem{Parameters}\begin{itemize}
\item {} 
\sphinxAtStartPar
\sphinxstyleliteralstrong{\sphinxupquote{S}} (\sphinxstyleliteralemphasis{\sphinxupquote{np.ndarray}}\sphinxstyleliteralemphasis{\sphinxupquote{{[}}}\sphinxstyleliteralemphasis{\sphinxupquote{float}}\sphinxstyleliteralemphasis{\sphinxupquote{{]}}}) \textendash{} Spatial covariance between all measured grid points plus the
covariance due to measurements (i.e. measurement noise, bias noise, and
sampling noise).

\item {} 
\sphinxAtStartPar
\sphinxstyleliteralstrong{\sphinxupquote{Ss}} (\sphinxstyleliteralemphasis{\sphinxupquote{np.ndarray}}\sphinxstyleliteralemphasis{\sphinxupquote{{[}}}\sphinxstyleliteralemphasis{\sphinxupquote{float}}\sphinxstyleliteralemphasis{\sphinxupquote{{]}}}) \textendash{} Covariance between the all (predicted) grid points and measured points.

\item {} 
\sphinxAtStartPar
\sphinxstyleliteralstrong{\sphinxupquote{grid\_obs}} (\sphinxstyleliteralemphasis{\sphinxupquote{np.ndarray}}\sphinxstyleliteralemphasis{\sphinxupquote{{[}}}\sphinxstyleliteralemphasis{\sphinxupquote{float}}\sphinxstyleliteralemphasis{\sphinxupquote{{]}}}) \textendash{} Gridded measurements (all measurement points averaged onto the output
gridboxes).

\item {} 
\sphinxAtStartPar
\sphinxstyleliteralstrong{\sphinxupquote{interp\_cov}} (\sphinxstyleliteralemphasis{\sphinxupquote{np.ndarray}}\sphinxstyleliteralemphasis{\sphinxupquote{{[}}}\sphinxstyleliteralemphasis{\sphinxupquote{float}}\sphinxstyleliteralemphasis{\sphinxupquote{{]}}}) \textendash{} interpolation covariance of all output grid points (each point in time
and all points against each other).

\item {} 
\sphinxAtStartPar
\sphinxstyleliteralstrong{\sphinxupquote{mean}} (\sphinxstyleliteralemphasis{\sphinxupquote{float}}) \textendash{} The constant mean of the output field.

\end{itemize}

\sphinxlineitem{Return type}
\sphinxAtStartPar
\DUrole{sphinx_autodoc_typehints-type}{\sphinxcode{\sphinxupquote{tuple}}{[}\sphinxcode{\sphinxupquote{ndarray}}, \sphinxcode{\sphinxupquote{ndarray}}{]}}

\sphinxlineitem{Returns}
\sphinxAtStartPar
\begin{itemize}
\item {} 
\sphinxAtStartPar
\sphinxstylestrong{z\_obs} (\sphinxstyleemphasis{np.ndarray{[}float{]}}) \textendash{} Full set of values for the whole domain derived from the observation
points using simple kriging.

\item {} 
\sphinxAtStartPar
\sphinxstylestrong{dz} (\sphinxstyleemphasis{np.ndarray{[}float{]}}) \textendash{} Uncertainty associated with the simple kriging.

\end{itemize}


\end{description}\end{quote}

\end{fulllineitems}

\index{unmasked\_kriging() (in module glomar\_gridding.kriging)@\spxentry{unmasked\_kriging()}\spxextra{in module glomar\_gridding.kriging}}

\begin{fulllineitems}
\phantomsection\label{\detokenize{users_guide:glomar_gridding.kriging.unmasked_kriging}}
\pysigstartsignatures
\pysiglinewithargsret
{\sphinxcode{\sphinxupquote{glomar\_gridding.kriging.}}\sphinxbfcode{\sphinxupquote{unmasked\_kriging}}}
{\sphinxparam{\DUrole{n}{unmask\_idx}}\sphinxparamcomma \sphinxparam{\DUrole{n}{unique\_obs\_idx}}\sphinxparamcomma \sphinxparam{\DUrole{n}{weights}}\sphinxparamcomma \sphinxparam{\DUrole{n}{obs}}\sphinxparamcomma \sphinxparam{\DUrole{n}{interp\_cov}}\sphinxparamcomma \sphinxparam{\DUrole{n}{error\_cov}}\sphinxparamcomma \sphinxparam{\DUrole{n}{remove\_obs\_mean}\DUrole{o}{=}\DUrole{default_value}{0}}\sphinxparamcomma \sphinxparam{\DUrole{n}{obs\_bias}\DUrole{o}{=}\DUrole{default_value}{None}}\sphinxparamcomma \sphinxparam{\DUrole{n}{method}\DUrole{o}{=}\DUrole{default_value}{\textquotesingle{}simple\textquotesingle{}}}}
{}
\pysigstopsignatures
\sphinxAtStartPar
Perform Kriging on a masked grid using a chosen method.

\sphinxAtStartPar
Get array of krigged observations and anomalies for all grid points in the
domain.
\begin{quote}\begin{description}
\sphinxlineitem{Parameters}\begin{itemize}
\item {} 
\sphinxAtStartPar
\sphinxstyleliteralstrong{\sphinxupquote{unmask\_idx}} (\sphinxstyleliteralemphasis{\sphinxupquote{np.ndarray}}\sphinxstyleliteralemphasis{\sphinxupquote{{[}}}\sphinxstyleliteralemphasis{\sphinxupquote{int}}\sphinxstyleliteralemphasis{\sphinxupquote{{]}}}) \textendash{} Indices of all un\sphinxhyphen{}masked points for chosen date.

\item {} 
\sphinxAtStartPar
\sphinxstyleliteralstrong{\sphinxupquote{unique\_obs\_idx}} (\sphinxstyleliteralemphasis{\sphinxupquote{np.ndarray}}\sphinxstyleliteralemphasis{\sphinxupquote{{[}}}\sphinxstyleliteralemphasis{\sphinxupquote{int}}\sphinxstyleliteralemphasis{\sphinxupquote{{]}}}) \textendash{} Unique indices of all measurement points for a chosen date,
representative of the indices of gridboxes, which have =\textgreater{} 1 measurement.

\item {} 
\sphinxAtStartPar
\sphinxstyleliteralstrong{\sphinxupquote{weights}} (\sphinxstyleliteralemphasis{\sphinxupquote{np.ndarray}}\sphinxstyleliteralemphasis{\sphinxupquote{{[}}}\sphinxstyleliteralemphasis{\sphinxupquote{float}}\sphinxstyleliteralemphasis{\sphinxupquote{{]}}}) \textendash{} Weight matrix (inverse of counts of observations).

\item {} 
\sphinxAtStartPar
\sphinxstyleliteralstrong{\sphinxupquote{obs}} (\sphinxstyleliteralemphasis{\sphinxupquote{np.ndarray}}\sphinxstyleliteralemphasis{\sphinxupquote{{[}}}\sphinxstyleliteralemphasis{\sphinxupquote{float}}\sphinxstyleliteralemphasis{\sphinxupquote{{]}}}) \textendash{} All point observations/measurements for the chosen date.

\item {} 
\sphinxAtStartPar
\sphinxstyleliteralstrong{\sphinxupquote{interp\_cov}} (\sphinxstyleliteralemphasis{\sphinxupquote{np.ndarray}}\sphinxstyleliteralemphasis{\sphinxupquote{{[}}}\sphinxstyleliteralemphasis{\sphinxupquote{float}}\sphinxstyleliteralemphasis{\sphinxupquote{{]}}}) \textendash{} Interpolation covariance of all output grid points (each point in time
and all points
against each other).

\item {} 
\sphinxAtStartPar
\sphinxstyleliteralstrong{\sphinxupquote{error\_cov}} (\sphinxstyleliteralemphasis{\sphinxupquote{np.ndarray}}\sphinxstyleliteralemphasis{\sphinxupquote{{[}}}\sphinxstyleliteralemphasis{\sphinxupquote{float}}\sphinxstyleliteralemphasis{\sphinxupquote{{]}}}) \textendash{} Measurement/Error covariance matrix.

\item {} 
\sphinxAtStartPar
\sphinxstyleliteralstrong{\sphinxupquote{remove\_obs\_mean}} (\sphinxstyleliteralemphasis{\sphinxupquote{int}}) \textendash{} Should the mean or median from obs be removed and added back onto obs?
0 = No (default action)
1 = the mean is removed
2 = the median is removed
3 = the spatial meam os removed

\item {} 
\sphinxAtStartPar
\sphinxstyleliteralstrong{\sphinxupquote{obs\_bias}} (\sphinxstyleliteralemphasis{\sphinxupquote{np.ndarray}}\sphinxstyleliteralemphasis{\sphinxupquote{{[}}}\sphinxstyleliteralemphasis{\sphinxupquote{float}}\sphinxstyleliteralemphasis{\sphinxupquote{{]} }}\sphinxstyleliteralemphasis{\sphinxupquote{| }}\sphinxstyleliteralemphasis{\sphinxupquote{None}}) \textendash{} Bias of all measurement points for a chosen date (corresponds to x\_obs).

\item {} 
\sphinxAtStartPar
\sphinxstyleliteralstrong{\sphinxupquote{method}} (\sphinxstyleliteralemphasis{\sphinxupquote{KrigMethod}}) \textendash{} The kriging method to use to fill in the output grid. One of “simple”
or “ordinary”.

\end{itemize}

\sphinxlineitem{Return type}
\sphinxAtStartPar
\DUrole{sphinx_autodoc_typehints-type}{\sphinxcode{\sphinxupquote{tuple}}{[}\sphinxcode{\sphinxupquote{ndarray}}, \sphinxcode{\sphinxupquote{ndarray}}{]}}

\sphinxlineitem{Returns}
\sphinxAtStartPar
\begin{itemize}
\item {} 
\sphinxAtStartPar
\sphinxstylestrong{z\_obs} (\sphinxstyleemphasis{np.ndarray{[}float{]}}) \textendash{} Full set of values for the whole domain derived from the observation
points using the chosen kriging method.

\item {} 
\sphinxAtStartPar
\sphinxstylestrong{dz} (\sphinxstyleemphasis{np.ndarray{[}float{]}}) \textendash{} Uncertainty associated with the chosen kriging method.

\end{itemize}


\end{description}\end{quote}

\end{fulllineitems}

\index{module@\spxentry{module}!glomar\_gridding.climatology@\spxentry{glomar\_gridding.climatology}}\index{glomar\_gridding.climatology@\spxentry{glomar\_gridding.climatology}!module@\spxentry{module}}

\section{Climatology}
\label{\detokenize{users_guide:climatology}}\label{\detokenize{users_guide:module-glomar_gridding.climatology}}
\sphinxAtStartPar
Functions for mapping climatologies and computing anomalies
\index{join\_climatology\_by\_doy() (in module glomar\_gridding.climatology)@\spxentry{join\_climatology\_by\_doy()}\spxextra{in module glomar\_gridding.climatology}}

\begin{fulllineitems}
\phantomsection\label{\detokenize{users_guide:glomar_gridding.climatology.join_climatology_by_doy}}
\pysigstartsignatures
\pysiglinewithargsret
{\sphinxcode{\sphinxupquote{glomar\_gridding.climatology.}}\sphinxbfcode{\sphinxupquote{join\_climatology\_by\_doy}}}
{\sphinxparam{\DUrole{n}{obs\_df}}\sphinxparamcomma \sphinxparam{\DUrole{n}{climatology\_365}}\sphinxparamcomma \sphinxparam{\DUrole{n}{lat\_col}\DUrole{o}{=}\DUrole{default_value}{\textquotesingle{}lat\textquotesingle{}}}\sphinxparamcomma \sphinxparam{\DUrole{n}{lon\_col}\DUrole{o}{=}\DUrole{default_value}{\textquotesingle{}lon\textquotesingle{}}}\sphinxparamcomma \sphinxparam{\DUrole{n}{date\_col}\DUrole{o}{=}\DUrole{default_value}{\textquotesingle{}date\textquotesingle{}}}\sphinxparamcomma \sphinxparam{\DUrole{n}{var\_col}\DUrole{o}{=}\DUrole{default_value}{\textquotesingle{}sst\textquotesingle{}}}\sphinxparamcomma \sphinxparam{\DUrole{n}{clim\_lat}\DUrole{o}{=}\DUrole{default_value}{\textquotesingle{}latitude\textquotesingle{}}}\sphinxparamcomma \sphinxparam{\DUrole{n}{clim\_lon}\DUrole{o}{=}\DUrole{default_value}{\textquotesingle{}longitude\textquotesingle{}}}\sphinxparamcomma \sphinxparam{\DUrole{n}{clim\_doy}\DUrole{o}{=}\DUrole{default_value}{\textquotesingle{}doy\textquotesingle{}}}\sphinxparamcomma \sphinxparam{\DUrole{n}{clim\_var}\DUrole{o}{=}\DUrole{default_value}{\textquotesingle{}climatology\textquotesingle{}}}\sphinxparamcomma \sphinxparam{\DUrole{n}{temp\_from\_kelvin}\DUrole{o}{=}\DUrole{default_value}{True}}}
{}
\pysigstopsignatures
\sphinxAtStartPar
Merge a climatology from an xarray.DataArray into a polars.DataFrame using
the day of year value and position.

\sphinxAtStartPar
This function accounts for leap years by taking the average of the
climatology values for 28th Feb and 1st March for observations that were
made on the 29th of Feb.

\sphinxAtStartPar
The climatology is merged into the DataFrame and anomaly values are
computed.
\begin{quote}\begin{description}
\sphinxlineitem{Parameters}\begin{itemize}
\item {} 
\sphinxAtStartPar
\sphinxstyleliteralstrong{\sphinxupquote{obs\_df}} (\sphinxstyleliteralemphasis{\sphinxupquote{polars.DataFrame}}) \textendash{} Observational DataFrame.

\item {} 
\sphinxAtStartPar
\sphinxstyleliteralstrong{\sphinxupquote{climatology\_365}} (\sphinxstyleliteralemphasis{\sphinxupquote{xarray.DataArray}}) \textendash{} DataArray containing daily climatology values (for 365 days).

\item {} 
\sphinxAtStartPar
\sphinxstyleliteralstrong{\sphinxupquote{lat\_col}} (\sphinxstyleliteralemphasis{\sphinxupquote{str}}) \textendash{} Name of the latitude column in the observational DataFrame.

\item {} 
\sphinxAtStartPar
\sphinxstyleliteralstrong{\sphinxupquote{lon\_col}} (\sphinxstyleliteralemphasis{\sphinxupquote{str}}) \textendash{} Name of the longitude column in the observational DataFrame.

\item {} 
\sphinxAtStartPar
\sphinxstyleliteralstrong{\sphinxupquote{date\_col}} (\sphinxstyleliteralemphasis{\sphinxupquote{str}}) \textendash{} Name of the datetime column in the observational DataFrame. Day of year
values are computed from this value.

\item {} 
\sphinxAtStartPar
\sphinxstyleliteralstrong{\sphinxupquote{var\_col}} (\sphinxstyleliteralemphasis{\sphinxupquote{str}}) \textendash{} Name of the variable column in the observational DataFrame. The merged
climatology names will have this name prefixed to “\_climatology”, the
anomaly values will have this name prefixed to “\_anomaly”.

\item {} 
\sphinxAtStartPar
\sphinxstyleliteralstrong{\sphinxupquote{clim\_lat}} (\sphinxstyleliteralemphasis{\sphinxupquote{str}}) \textendash{} Name of the latitude coordinate in the climatology DataArray.

\item {} 
\sphinxAtStartPar
\sphinxstyleliteralstrong{\sphinxupquote{clim\_lon}} (\sphinxstyleliteralemphasis{\sphinxupquote{str}}) \textendash{} Name of the longitude coordinate in the climatology DataArray.

\item {} 
\sphinxAtStartPar
\sphinxstyleliteralstrong{\sphinxupquote{clim\_doy}} (\sphinxstyleliteralemphasis{\sphinxupquote{str}}) \textendash{} Name of the day of year coordinate in the climatology DataArray.

\item {} 
\sphinxAtStartPar
\sphinxstyleliteralstrong{\sphinxupquote{clim\_var}} (\sphinxstyleliteralemphasis{\sphinxupquote{str}}) \textendash{} Name of the climatology variable in the climatology DataArray.

\item {} 
\sphinxAtStartPar
\sphinxstyleliteralstrong{\sphinxupquote{temp\_from\_kelvin}} (\sphinxstyleliteralemphasis{\sphinxupquote{bool}}) \textendash{} Optionally adjust the climatology from Kelvin to Celcius if the variable
is a temperature.

\end{itemize}

\sphinxlineitem{Returns}
\sphinxAtStartPar
\sphinxstylestrong{obs\_df} \textendash{} With the climatology merged and anomaly computed. The new columns are
“\_climatology” and “\_anomaly” prefixed by the \sphinxtitleref{var\_col} value
respectively.

\sphinxlineitem{Return type}
\sphinxAtStartPar
polars.DataFrame

\end{description}\end{quote}

\end{fulllineitems}

\index{read\_climatology() (in module glomar\_gridding.climatology)@\spxentry{read\_climatology()}\spxextra{in module glomar\_gridding.climatology}}

\begin{fulllineitems}
\phantomsection\label{\detokenize{users_guide:glomar_gridding.climatology.read_climatology}}
\pysigstartsignatures
\pysiglinewithargsret
{\sphinxcode{\sphinxupquote{glomar\_gridding.climatology.}}\sphinxbfcode{\sphinxupquote{read\_climatology}}}
{\sphinxparam{\DUrole{n}{clim\_path}}\sphinxparamcomma \sphinxparam{\DUrole{n}{min\_lat}\DUrole{o}{=}\DUrole{default_value}{\sphinxhyphen{}90}}\sphinxparamcomma \sphinxparam{\DUrole{n}{max\_lat}\DUrole{o}{=}\DUrole{default_value}{90}}\sphinxparamcomma \sphinxparam{\DUrole{n}{min\_lon}\DUrole{o}{=}\DUrole{default_value}{\sphinxhyphen{}180}}\sphinxparamcomma \sphinxparam{\DUrole{n}{max\_lon}\DUrole{o}{=}\DUrole{default_value}{180}}\sphinxparamcomma \sphinxparam{\DUrole{n}{lat\_var}\DUrole{o}{=}\DUrole{default_value}{\textquotesingle{}lat\textquotesingle{}}}\sphinxparamcomma \sphinxparam{\DUrole{n}{lon\_var}\DUrole{o}{=}\DUrole{default_value}{\textquotesingle{}lon\textquotesingle{}}}\sphinxparamcomma \sphinxparam{\DUrole{o}{**}\DUrole{n}{kwargs}}}
{}
\pysigstopsignatures
\sphinxAtStartPar
Load a climatology dataset from a netCDF file.
\begin{quote}\begin{description}
\sphinxlineitem{Parameters}\begin{itemize}
\item {} 
\sphinxAtStartPar
\sphinxstyleliteralstrong{\sphinxupquote{clim\_path}} (\sphinxstyleliteralemphasis{\sphinxupquote{str}}) \textendash{} Path to the climatology file. Can contain format blocks to be replaced
by the values passed to kwargs.

\item {} 
\sphinxAtStartPar
\sphinxstyleliteralstrong{\sphinxupquote{min\_lat}} (\sphinxstyleliteralemphasis{\sphinxupquote{float}}) \textendash{} Minimum latitude to load.

\item {} 
\sphinxAtStartPar
\sphinxstyleliteralstrong{\sphinxupquote{max\_lat}} (\sphinxstyleliteralemphasis{\sphinxupquote{float}}) \textendash{} Maximum latitude to load.

\item {} 
\sphinxAtStartPar
\sphinxstyleliteralstrong{\sphinxupquote{min\_lon}} (\sphinxstyleliteralemphasis{\sphinxupquote{float}}) \textendash{} Minimum longitude to load.

\item {} 
\sphinxAtStartPar
\sphinxstyleliteralstrong{\sphinxupquote{max\_lon}} (\sphinxstyleliteralemphasis{\sphinxupquote{float}}) \textendash{} Maximum longitude to load.

\item {} 
\sphinxAtStartPar
\sphinxstyleliteralstrong{\sphinxupquote{lat\_var}} (\sphinxstyleliteralemphasis{\sphinxupquote{str}}) \textendash{} Name of the latitude variable.

\item {} 
\sphinxAtStartPar
\sphinxstyleliteralstrong{\sphinxupquote{lon\_var}} (\sphinxstyleliteralemphasis{\sphinxupquote{str}}) \textendash{} Name of the longitude variable.

\item {} 
\sphinxAtStartPar
\sphinxstyleliteralstrong{\sphinxupquote{**kwargs}} \textendash{} Replacement values for the climatology path.

\end{itemize}

\sphinxlineitem{Returns}
\sphinxAtStartPar
\sphinxstylestrong{clim\_ds} \textendash{} Containing the climatology bounded by the min/max arguments provided.

\sphinxlineitem{Return type}
\sphinxAtStartPar
xarray.Dataset

\end{description}\end{quote}

\end{fulllineitems}

\index{module@\spxentry{module}!glomar\_gridding.mask@\spxentry{glomar\_gridding.mask}}\index{glomar\_gridding.mask@\spxentry{glomar\_gridding.mask}!module@\spxentry{module}}

\section{Masking}
\label{\detokenize{users_guide:masking}}\label{\detokenize{users_guide:module-glomar_gridding.mask}}
\sphinxAtStartPar
Functions for applying masks to grids and DataFrames
\index{get\_mask\_idx() (in module glomar\_gridding.mask)@\spxentry{get\_mask\_idx()}\spxextra{in module glomar\_gridding.mask}}

\begin{fulllineitems}
\phantomsection\label{\detokenize{users_guide:glomar_gridding.mask.get_mask_idx}}
\pysigstartsignatures
\pysiglinewithargsret
{\sphinxcode{\sphinxupquote{glomar\_gridding.mask.}}\sphinxbfcode{\sphinxupquote{get\_mask\_idx}}}
{\sphinxparam{\DUrole{n}{mask}}\sphinxparamcomma \sphinxparam{\DUrole{n}{mask\_val}\DUrole{o}{=}\DUrole{default_value}{nan}}\sphinxparamcomma \sphinxparam{\DUrole{n}{masked}\DUrole{o}{=}\DUrole{default_value}{True}}}
{}
\pysigstopsignatures
\sphinxAtStartPar
Get the 1d indices of masked values from a mask array.
\begin{quote}\begin{description}
\sphinxlineitem{Parameters}\begin{itemize}
\item {} 
\sphinxAtStartPar
\sphinxstyleliteralstrong{\sphinxupquote{mask}} (\sphinxstyleliteralemphasis{\sphinxupquote{xarray.DataArray}}) \textendash{} The mask array, containing values indicated a masked value.

\item {} 
\sphinxAtStartPar
\sphinxstyleliteralstrong{\sphinxupquote{mask\_val}} (\sphinxstyleliteralemphasis{\sphinxupquote{Any}}) \textendash{} The value that indicates the position should be masked.

\item {} 
\sphinxAtStartPar
\sphinxstyleliteralstrong{\sphinxupquote{masked}} (\sphinxstyleliteralemphasis{\sphinxupquote{bool}}) \textendash{} Return indices where values in the mask array equal this value. If set
to False it will return indices where values are not equal to the mask
value. Can be used to get unmasked indices if this value is set to
False.

\end{itemize}

\sphinxlineitem{Return type}
\sphinxAtStartPar
An array of integers indicating the indices which are masked.

\end{description}\end{quote}

\end{fulllineitems}

\index{mask\_array() (in module glomar\_gridding.mask)@\spxentry{mask\_array()}\spxextra{in module glomar\_gridding.mask}}

\begin{fulllineitems}
\phantomsection\label{\detokenize{users_guide:glomar_gridding.mask.mask_array}}
\pysigstartsignatures
\pysiglinewithargsret
{\sphinxcode{\sphinxupquote{glomar\_gridding.mask.}}\sphinxbfcode{\sphinxupquote{mask\_array}}}
{\sphinxparam{\DUrole{n}{grid}}\sphinxparamcomma \sphinxparam{\DUrole{n}{mask}}\sphinxparamcomma \sphinxparam{\DUrole{n}{varname}}\sphinxparamcomma \sphinxparam{\DUrole{n}{masked\_value}\DUrole{o}{=}\DUrole{default_value}{nan}}\sphinxparamcomma \sphinxparam{\DUrole{n}{mask\_value}\DUrole{o}{=}\DUrole{default_value}{True}}}
{}
\pysigstopsignatures
\sphinxAtStartPar
Apply a mask to a DataArray.

\sphinxAtStartPar
The grid and mask must already align for this function to work. An error
will be raised if the coordinate systems cannot be aligned.
\begin{quote}\begin{description}
\sphinxlineitem{Parameters}\begin{itemize}
\item {} 
\sphinxAtStartPar
\sphinxstyleliteralstrong{\sphinxupquote{grid}} (\sphinxstyleliteralemphasis{\sphinxupquote{xarray.DataArray}}) \textendash{} Observational DataArray to be masked by postitions in the mask
DataArray.

\item {} 
\sphinxAtStartPar
\sphinxstyleliteralstrong{\sphinxupquote{mask}} (\sphinxstyleliteralemphasis{\sphinxupquote{xarray.DataArray}}) \textendash{} Array containing vlaues used to mask the observational DataFrame.

\item {} 
\sphinxAtStartPar
\sphinxstyleliteralstrong{\sphinxupquote{varname}} (\sphinxstyleliteralemphasis{\sphinxupquote{str}}) \textendash{} Name of the variable in the observational DataArray to apply the mask
to.

\item {} 
\sphinxAtStartPar
\sphinxstyleliteralstrong{\sphinxupquote{masked\_value}} (\sphinxstyleliteralemphasis{\sphinxupquote{Any}}) \textendash{} Value indicating masked values in the DataArray.

\item {} 
\sphinxAtStartPar
\sphinxstyleliteralstrong{\sphinxupquote{mask\_value}} (\sphinxstyleliteralemphasis{\sphinxupquote{Any}}) \textendash{} Value to set masked values to in the observational DataFrame.

\end{itemize}

\sphinxlineitem{Returns}
\sphinxAtStartPar
\sphinxstylestrong{grid} \textendash{} Input xarray.DataArray with the variable masked by the mask DataArray.

\sphinxlineitem{Return type}
\sphinxAtStartPar
xarray.DataArray

\end{description}\end{quote}

\end{fulllineitems}

\index{mask\_dataset() (in module glomar\_gridding.mask)@\spxentry{mask\_dataset()}\spxextra{in module glomar\_gridding.mask}}

\begin{fulllineitems}
\phantomsection\label{\detokenize{users_guide:glomar_gridding.mask.mask_dataset}}
\pysigstartsignatures
\pysiglinewithargsret
{\sphinxcode{\sphinxupquote{glomar\_gridding.mask.}}\sphinxbfcode{\sphinxupquote{mask\_dataset}}}
{\sphinxparam{\DUrole{n}{dataset}}\sphinxparamcomma \sphinxparam{\DUrole{n}{mask}}\sphinxparamcomma \sphinxparam{\DUrole{n}{varnames}}\sphinxparamcomma \sphinxparam{\DUrole{n}{masked\_value}\DUrole{o}{=}\DUrole{default_value}{nan}}\sphinxparamcomma \sphinxparam{\DUrole{n}{mask\_value}\DUrole{o}{=}\DUrole{default_value}{True}}}
{}
\pysigstopsignatures
\sphinxAtStartPar
Apply a mask to a DataSet.

\sphinxAtStartPar
The grid and mask must already align for this function to work. An error
will be raised if the coordinate systems cannot be aligned.
\begin{quote}\begin{description}
\sphinxlineitem{Parameters}\begin{itemize}
\item {} 
\sphinxAtStartPar
\sphinxstyleliteralstrong{\sphinxupquote{dataset}} (\sphinxstyleliteralemphasis{\sphinxupquote{xarray.Dataset}}) \textendash{} Observational Dataset to be masked by postitions in the mask
DataArray.

\item {} 
\sphinxAtStartPar
\sphinxstyleliteralstrong{\sphinxupquote{mask}} (\sphinxstyleliteralemphasis{\sphinxupquote{xarray.DataArray}}) \textendash{} Array containing vlaues used to mask the observational DataFrame.

\item {} 
\sphinxAtStartPar
\sphinxstyleliteralstrong{\sphinxupquote{varnames}} (\sphinxstyleliteralemphasis{\sphinxupquote{str}}\sphinxstyleliteralemphasis{\sphinxupquote{ | }}\sphinxstyleliteralemphasis{\sphinxupquote{list}}\sphinxstyleliteralemphasis{\sphinxupquote{{[}}}\sphinxstyleliteralemphasis{\sphinxupquote{str}}\sphinxstyleliteralemphasis{\sphinxupquote{{]}}}) \textendash{} A list containing the names of  variables in the observational Dataser
to apply the mask to.

\item {} 
\sphinxAtStartPar
\sphinxstyleliteralstrong{\sphinxupquote{masked\_value}} (\sphinxstyleliteralemphasis{\sphinxupquote{Any}}) \textendash{} Value indicating masked values in the DataArray.

\item {} 
\sphinxAtStartPar
\sphinxstyleliteralstrong{\sphinxupquote{mask\_value}} (\sphinxstyleliteralemphasis{\sphinxupquote{Any}}) \textendash{} Value to set masked values to in the observational DataFrame.

\end{itemize}

\sphinxlineitem{Returns}
\sphinxAtStartPar
\sphinxstylestrong{grid} \textendash{} Input xarray.Dataset with the variables masked by the mask DataArray.

\sphinxlineitem{Return type}
\sphinxAtStartPar
xarray.Dataset

\end{description}\end{quote}

\end{fulllineitems}

\index{mask\_from\_obs\_array() (in module glomar\_gridding.mask)@\spxentry{mask\_from\_obs\_array()}\spxextra{in module glomar\_gridding.mask}}

\begin{fulllineitems}
\phantomsection\label{\detokenize{users_guide:glomar_gridding.mask.mask_from_obs_array}}
\pysigstartsignatures
\pysiglinewithargsret
{\sphinxcode{\sphinxupquote{glomar\_gridding.mask.}}\sphinxbfcode{\sphinxupquote{mask\_from\_obs\_array}}}
{\sphinxparam{\DUrole{n}{obs}}\sphinxparamcomma \sphinxparam{\DUrole{n}{datetime\_idx}}}
{}
\pysigstopsignatures
\sphinxAtStartPar
Infer a mask from an input array. Mask values are those where all values
are NaN along the time dimension.

\sphinxAtStartPar
An example use\sphinxhyphen{}case would be to infer land\sphinxhyphen{}points from a SST data array.
\begin{quote}\begin{description}
\sphinxlineitem{Parameters}\begin{itemize}
\item {} 
\sphinxAtStartPar
\sphinxstyleliteralstrong{\sphinxupquote{obs}} (\sphinxstyleliteralemphasis{\sphinxupquote{numpy.ndarray}}) \textendash{} Array containing the observation values. Records that are numpy.nan
will count towards the mask, if all values in the datetime dimension
are numpy.nan.

\item {} 
\sphinxAtStartPar
\sphinxstyleliteralstrong{\sphinxupquote{datetime\_idx}} (\sphinxstyleliteralemphasis{\sphinxupquote{int}}) \textendash{} The index of the datetime, or grouping, dimension. If all records at
a point along this dimension are NaN then this point will be masked.

\end{itemize}

\sphinxlineitem{Returns}
\sphinxAtStartPar
\sphinxstylestrong{mask} \textendash{} A boolean array with dimension reduced along the datetime dimension.
A True value indicates that all values along the datetime dimension
for this index are numpy.nan and are masked.

\sphinxlineitem{Return type}
\sphinxAtStartPar
numpy.ndarray

\end{description}\end{quote}

\end{fulllineitems}

\index{mask\_from\_obs\_frame() (in module glomar\_gridding.mask)@\spxentry{mask\_from\_obs\_frame()}\spxextra{in module glomar\_gridding.mask}}

\begin{fulllineitems}
\phantomsection\label{\detokenize{users_guide:glomar_gridding.mask.mask_from_obs_frame}}
\pysigstartsignatures
\pysiglinewithargsret
{\sphinxcode{\sphinxupquote{glomar\_gridding.mask.}}\sphinxbfcode{\sphinxupquote{mask\_from\_obs\_frame}}}
{\sphinxparam{\DUrole{n}{obs}}\sphinxparamcomma \sphinxparam{\DUrole{n}{coords}}\sphinxparamcomma \sphinxparam{\DUrole{n}{datetime\_col}}\sphinxparamcomma \sphinxparam{\DUrole{n}{value\_col}}}
{}
\pysigstopsignatures
\sphinxAtStartPar
Compute a mask from observations.

\sphinxAtStartPar
Positions defined by the “coords” values that do not have any observations,
at any datetime value in the “datetime\_col”, for the “value\_col” field are
masked.

\sphinxAtStartPar
An example use\sphinxhyphen{}case would be to identify land positions from sst records.
\begin{quote}\begin{description}
\sphinxlineitem{Parameters}\begin{itemize}
\item {} 
\sphinxAtStartPar
\sphinxstyleliteralstrong{\sphinxupquote{obs}} (\sphinxstyleliteralemphasis{\sphinxupquote{polars.DataFrame}}) \textendash{} DataFrame constaining observations over space and time. The values in
the “value\_col” field will be used to define the mask.

\item {} 
\sphinxAtStartPar
\sphinxstyleliteralstrong{\sphinxupquote{coords}} (\sphinxstyleliteralemphasis{\sphinxupquote{str}}\sphinxstyleliteralemphasis{\sphinxupquote{ | }}\sphinxstyleliteralemphasis{\sphinxupquote{list}}\sphinxstyleliteralemphasis{\sphinxupquote{{[}}}\sphinxstyleliteralemphasis{\sphinxupquote{str}}\sphinxstyleliteralemphasis{\sphinxupquote{{]}}}) \textendash{} A list of columns containing the coordinates used to define the mask.
For example {[}“lat”, “lon”{]}.

\item {} 
\sphinxAtStartPar
\sphinxstyleliteralstrong{\sphinxupquote{datetime\_col}} (\sphinxstyleliteralemphasis{\sphinxupquote{str}}) \textendash{} Name of the datetime column. Any positions that contain no records at
any datetime value are masked.

\item {} 
\sphinxAtStartPar
\sphinxstyleliteralstrong{\sphinxupquote{value\_col}} (\sphinxstyleliteralemphasis{\sphinxupquote{str}}) \textendash{} Name of the column containing values from which the mask will be
defined.

\end{itemize}

\sphinxlineitem{Return type}
\sphinxAtStartPar
\DUrole{sphinx_autodoc_typehints-type}{\sphinxcode{\sphinxupquote{DataFrame}}}

\sphinxlineitem{Returns}
\sphinxAtStartPar
\begin{itemize}
\item {} 
\sphinxAtStartPar
\sphinxstyleemphasis{polars.DataFrame containing coordinate columns and a Boolean “mask” column}

\item {} 
\sphinxAtStartPar
\sphinxstyleemphasis{indicating positions that contain no observations and would be a mask value.}

\end{itemize}


\end{description}\end{quote}

\end{fulllineitems}

\index{mask\_observations() (in module glomar\_gridding.mask)@\spxentry{mask\_observations()}\spxextra{in module glomar\_gridding.mask}}

\begin{fulllineitems}
\phantomsection\label{\detokenize{users_guide:glomar_gridding.mask.mask_observations}}
\pysigstartsignatures
\pysiglinewithargsret
{\sphinxcode{\sphinxupquote{glomar\_gridding.mask.}}\sphinxbfcode{\sphinxupquote{mask\_observations}}}
{\sphinxparam{\DUrole{n}{obs}}\sphinxparamcomma \sphinxparam{\DUrole{n}{mask}}\sphinxparamcomma \sphinxparam{\DUrole{n}{varnames}}\sphinxparamcomma \sphinxparam{\DUrole{n}{mask\_varname}\DUrole{o}{=}\DUrole{default_value}{\textquotesingle{}mask\textquotesingle{}}}\sphinxparamcomma \sphinxparam{\DUrole{n}{masked\_value}\DUrole{o}{=}\DUrole{default_value}{nan}}\sphinxparamcomma \sphinxparam{\DUrole{n}{mask\_value}\DUrole{o}{=}\DUrole{default_value}{True}}\sphinxparamcomma \sphinxparam{\DUrole{n}{obs\_coords}\DUrole{o}{=}\DUrole{default_value}{{[}\textquotesingle{}lat\textquotesingle{}, \textquotesingle{}lon\textquotesingle{}{]}}}\sphinxparamcomma \sphinxparam{\DUrole{n}{mask\_coords}\DUrole{o}{=}\DUrole{default_value}{{[}\textquotesingle{}latitude\textquotesingle{}, \textquotesingle{}longitude\textquotesingle{}{]}}}\sphinxparamcomma \sphinxparam{\DUrole{n}{align\_to\_mask}\DUrole{o}{=}\DUrole{default_value}{False}}\sphinxparamcomma \sphinxparam{\DUrole{n}{drop}\DUrole{o}{=}\DUrole{default_value}{False}}\sphinxparamcomma \sphinxparam{\DUrole{n}{mask\_grid\_prefix}\DUrole{o}{=}\DUrole{default_value}{\textquotesingle{}\_mask\_grid\_\textquotesingle{}}}}
{}
\pysigstopsignatures
\sphinxAtStartPar
Mask observations in a DataFrame subject to a mask DataArray.
\begin{quote}\begin{description}
\sphinxlineitem{Parameters}\begin{itemize}
\item {} 
\sphinxAtStartPar
\sphinxstyleliteralstrong{\sphinxupquote{obs}} (\sphinxstyleliteralemphasis{\sphinxupquote{polars.DataFrame}}) \textendash{} Observational DataFrame to be masked by postitions in the mask
DataArray.

\item {} 
\sphinxAtStartPar
\sphinxstyleliteralstrong{\sphinxupquote{mask}} (\sphinxstyleliteralemphasis{\sphinxupquote{xarray.DataArray}}) \textendash{} Array containing vlaues used to mask the observational DataFrame.

\item {} 
\sphinxAtStartPar
\sphinxstyleliteralstrong{\sphinxupquote{varnames}} (\sphinxstyleliteralemphasis{\sphinxupquote{str}}\sphinxstyleliteralemphasis{\sphinxupquote{ | }}\sphinxstyleliteralemphasis{\sphinxupquote{list}}\sphinxstyleliteralemphasis{\sphinxupquote{{[}}}\sphinxstyleliteralemphasis{\sphinxupquote{str}}\sphinxstyleliteralemphasis{\sphinxupquote{{]}}}) \textendash{} Columns in the observational DataFrame to apply the mask to.

\item {} 
\sphinxAtStartPar
\sphinxstyleliteralstrong{\sphinxupquote{mask\_varname}} (\sphinxstyleliteralemphasis{\sphinxupquote{str}}) \textendash{} Name of the mask variable in the mask DataArray.

\item {} 
\sphinxAtStartPar
\sphinxstyleliteralstrong{\sphinxupquote{masked\_value}} (\sphinxstyleliteralemphasis{\sphinxupquote{Any}}) \textendash{} Value indicating masked values in the DataArray.

\item {} 
\sphinxAtStartPar
\sphinxstyleliteralstrong{\sphinxupquote{mask\_value}} (\sphinxstyleliteralemphasis{\sphinxupquote{Any}}) \textendash{} Value to set masked values to in the observational DataFrame.

\item {} 
\sphinxAtStartPar
\sphinxstyleliteralstrong{\sphinxupquote{obs\_coords}} (\sphinxstyleliteralemphasis{\sphinxupquote{list}}\sphinxstyleliteralemphasis{\sphinxupquote{{[}}}\sphinxstyleliteralemphasis{\sphinxupquote{str}}\sphinxstyleliteralemphasis{\sphinxupquote{{]}}}) \textendash{} A list of coordinate names in the observational DataFrame. Used to map
the mask DataArray to the observational DataFrame. The order must align
with the coordinates of the mask DataArray.

\item {} 
\sphinxAtStartPar
\sphinxstyleliteralstrong{\sphinxupquote{mask\_coords}} (\sphinxstyleliteralemphasis{\sphinxupquote{list}}\sphinxstyleliteralemphasis{\sphinxupquote{{[}}}\sphinxstyleliteralemphasis{\sphinxupquote{str}}\sphinxstyleliteralemphasis{\sphinxupquote{{]}}}) \textendash{} A list of coordinate names in the mask DataArray. These coordinates are
mapped onto the observational DataFrame in order to apply the mask. The
ordering of the coordinate names in this list must match those in the
obs\_coords list.

\item {} 
\sphinxAtStartPar
\sphinxstyleliteralstrong{\sphinxupquote{align\_to\_mask}} (\sphinxstyleliteralemphasis{\sphinxupquote{bool}}) \textendash{} Optionally align the observational DataFrame to the mask DataArray.
This essentially sets the mask’s grid as the output grid for
interpolation.

\item {} 
\sphinxAtStartPar
\sphinxstyleliteralstrong{\sphinxupquote{drop}} (\sphinxstyleliteralemphasis{\sphinxupquote{bool}}) \textendash{} Drop masked values in the observational DataFrame.

\item {} 
\sphinxAtStartPar
\sphinxstyleliteralstrong{\sphinxupquote{mask\_grid\_prefix}} (\sphinxstyleliteralemphasis{\sphinxupquote{str}}) \textendash{} Prefix to use for the mask gridbox index column in the observational
DataFrame.

\end{itemize}

\sphinxlineitem{Returns}
\sphinxAtStartPar
\sphinxstylestrong{obs} \textendash{} Input polars.DataFrame containing additional column named by the
mask\_varname argument, indicating records that are masked. Masked values
are dropped if the drop argument is set to True.

\sphinxlineitem{Return type}
\sphinxAtStartPar
polars.DataFrame

\end{description}\end{quote}

\end{fulllineitems}

\index{module@\spxentry{module}!glomar\_gridding.distances@\spxentry{glomar\_gridding.distances}}\index{glomar\_gridding.distances@\spxentry{glomar\_gridding.distances}!module@\spxentry{module}}

\section{Distances}
\label{\detokenize{users_guide:distances}}\label{\detokenize{users_guide:module-glomar_gridding.distances}}
\sphinxAtStartPar
Functions for calculating distances or distance\sphinxhyphen{}based covariance components.

\sphinxAtStartPar
Some functions can be used for computing pairwise\sphinxhyphen{}distances, for example via
squareform. Some functions can be used as a distance function for
glomar\_gridding.error\_covariance.dist\_weights, accounting for the distance
component to an error covariance matrix.

\sphinxAtStartPar
Functions for computing covariance using Matern Tau by Steven Chan (@stchan).
\index{calculate\_distance\_matrix() (in module glomar\_gridding.distances)@\spxentry{calculate\_distance\_matrix()}\spxextra{in module glomar\_gridding.distances}}

\begin{fulllineitems}
\phantomsection\label{\detokenize{users_guide:glomar_gridding.distances.calculate_distance_matrix}}
\pysigstartsignatures
\pysiglinewithargsret
{\sphinxcode{\sphinxupquote{glomar\_gridding.distances.}}\sphinxbfcode{\sphinxupquote{calculate\_distance\_matrix}}}
{\sphinxparam{\DUrole{n}{df}}\sphinxparamcomma \sphinxparam{\DUrole{n}{dist\_func=\textless{}function haversine\_distance\textgreater{}}}\sphinxparamcomma \sphinxparam{\DUrole{n}{lat\_col=\textquotesingle{}lat\textquotesingle{}}}\sphinxparamcomma \sphinxparam{\DUrole{n}{lon\_col=\textquotesingle{}lon\textquotesingle{}}}}
{}
\pysigstopsignatures
\sphinxAtStartPar
Create a distance matrix from a DataFrame containing positional information,
typically latitude and longitude, using a distance function.

\sphinxAtStartPar
Available functions are \sphinxtitleref{haversine\_distance}, \sphinxtitleref{euclidean\_distance}. A
custom function can be used, requiring that the function takes the form:
(tuple{[}float, float{]}, tuple{[}float, float{]}) \sphinxhyphen{}\textgreater{} float
\begin{quote}\begin{description}
\sphinxlineitem{Parameters}\begin{itemize}
\item {} 
\sphinxAtStartPar
\sphinxstyleliteralstrong{\sphinxupquote{df}} (\sphinxstyleliteralemphasis{\sphinxupquote{polars.DataFrame}}) \textendash{} DataFrame containing latitude and longitude columns indicating the
positions between which distances are computed to form the distance
matrix

\item {} 
\sphinxAtStartPar
\sphinxstyleliteralstrong{\sphinxupquote{dist\_func}} (\sphinxstyleliteralemphasis{\sphinxupquote{Callable}}) \textendash{} The function used to calculate the pairwise distances. Functions
available for this function are \sphinxtitleref{haversine\_distance} and
\sphinxtitleref{euclidean\_distance}.
A custom function can be based, that takes as input two tuples of
positions (computing a single distance value between the pair of
positions). (tuple{[}float, float{]}, tuple{[}float, float{]}) \sphinxhyphen{}\textgreater{} float

\item {} 
\sphinxAtStartPar
\sphinxstyleliteralstrong{\sphinxupquote{lat\_col}} (\sphinxstyleliteralemphasis{\sphinxupquote{str}}) \textendash{} Name of the column in the input DataFrame containing latitude values.

\item {} 
\sphinxAtStartPar
\sphinxstyleliteralstrong{\sphinxupquote{lon\_col}} (\sphinxstyleliteralemphasis{\sphinxupquote{str}}) \textendash{} Name of the column in the input DataFrame containing longitude values.

\end{itemize}

\sphinxlineitem{Returns}
\sphinxAtStartPar
\sphinxstylestrong{dist} \textendash{} A matrix of pairwise distances.

\sphinxlineitem{Return type}
\sphinxAtStartPar
np.ndarray{[}float{]}

\end{description}\end{quote}

\end{fulllineitems}

\index{euclidean\_distance() (in module glomar\_gridding.distances)@\spxentry{euclidean\_distance()}\spxextra{in module glomar\_gridding.distances}}

\begin{fulllineitems}
\phantomsection\label{\detokenize{users_guide:glomar_gridding.distances.euclidean_distance}}
\pysigstartsignatures
\pysiglinewithargsret
{\sphinxcode{\sphinxupquote{glomar\_gridding.distances.}}\sphinxbfcode{\sphinxupquote{euclidean\_distance}}}
{\sphinxparam{\DUrole{n}{df}}\sphinxparamcomma \sphinxparam{\DUrole{n}{radius}\DUrole{o}{=}\DUrole{default_value}{6371.0}}}
{}
\pysigstopsignatures
\sphinxAtStartPar
Calculate the Euclidean distance in kilometers between pairs of lat, lon
points on the earth (specified in decimal degrees).

\sphinxAtStartPar
See:
\sphinxurl{https://math.stackexchange.com/questions/29157/how-do-i-convert-the-distance-between-two-lat-long-points-into-feet-meters}
\sphinxurl{https://cesar.esa.int/upload/201709/Earth\_Coordinates\_Booklet.pdf}

\sphinxAtStartPar
d = SQRT((x\_2\sphinxhyphen{}x\_1)**2 + (y\_2\sphinxhyphen{}y\_1)**2 + (z\_2\sphinxhyphen{}z\_1)**2)

\sphinxAtStartPar
where

\sphinxAtStartPar
(x\_n y\_n z\_n) = ( Rcos(lat)cos(lon) Rcos(lat)sin(lon) Rsin(lat) )
\begin{quote}\begin{description}
\sphinxlineitem{Parameters}\begin{itemize}
\item {} 
\sphinxAtStartPar
\sphinxstyleliteralstrong{\sphinxupquote{df}} (\sphinxstyleliteralemphasis{\sphinxupquote{polars.DataFrame}}) \textendash{} DataFrame containing latitude and longitude columns indicating the
positions between which distances are computed to form the distance
matrix

\item {} 
\sphinxAtStartPar
\sphinxstyleliteralstrong{\sphinxupquote{radius}} (\sphinxstyleliteralemphasis{\sphinxupquote{float}}) \textendash{} The radius of the sphere used for the calculation. Defaults to the
radius of the earth in km (6371.0 km).

\end{itemize}

\sphinxlineitem{Returns}
\sphinxAtStartPar
\sphinxstylestrong{dist} \textendash{} The direct pairwise distance between the positions in the input
DataFrame through the sphere defined by the radius parameter.

\sphinxlineitem{Return type}
\sphinxAtStartPar
float

\end{description}\end{quote}

\end{fulllineitems}

\index{haversine\_distance() (in module glomar\_gridding.distances)@\spxentry{haversine\_distance()}\spxextra{in module glomar\_gridding.distances}}

\begin{fulllineitems}
\phantomsection\label{\detokenize{users_guide:glomar_gridding.distances.haversine_distance}}
\pysigstartsignatures
\pysiglinewithargsret
{\sphinxcode{\sphinxupquote{glomar\_gridding.distances.}}\sphinxbfcode{\sphinxupquote{haversine\_distance}}}
{\sphinxparam{\DUrole{n}{df}}\sphinxparamcomma \sphinxparam{\DUrole{n}{radius}\DUrole{o}{=}\DUrole{default_value}{6371}}}
{}
\pysigstopsignatures
\sphinxAtStartPar
Calculate the great circle distance in kilometers between pairs of lat, lon
points on the earth (specified in decimal degrees).
\begin{quote}\begin{description}
\sphinxlineitem{Parameters}\begin{itemize}
\item {} 
\sphinxAtStartPar
\sphinxstyleliteralstrong{\sphinxupquote{df}} (\sphinxstyleliteralemphasis{\sphinxupquote{polars.DataFrame}}) \textendash{} DataFrame containing latitude and longitude columns indicating the
positions between which distances are computed to form the distance
matrix

\item {} 
\sphinxAtStartPar
\sphinxstyleliteralstrong{\sphinxupquote{radius}} (\sphinxstyleliteralemphasis{\sphinxupquote{float}}) \textendash{} The radius of the sphere used for the calculation. Defaults to the
radius of the earth in km (6371.0 km).

\end{itemize}

\sphinxlineitem{Returns}
\sphinxAtStartPar
\sphinxstylestrong{dist} \textendash{} The pairwise haversine distances between the inputs in the DataFrame,
on the sphere defined by the radius parameter.

\sphinxlineitem{Return type}
\sphinxAtStartPar
numpy.ndarray

\end{description}\end{quote}

\end{fulllineitems}

\index{haversine\_gaussian() (in module glomar\_gridding.distances)@\spxentry{haversine\_gaussian()}\spxextra{in module glomar\_gridding.distances}}

\begin{fulllineitems}
\phantomsection\label{\detokenize{users_guide:glomar_gridding.distances.haversine_gaussian}}
\pysigstartsignatures
\pysiglinewithargsret
{\sphinxcode{\sphinxupquote{glomar\_gridding.distances.}}\sphinxbfcode{\sphinxupquote{haversine\_gaussian}}}
{\sphinxparam{\DUrole{n}{df}}\sphinxparamcomma \sphinxparam{\DUrole{n}{R}\DUrole{o}{=}\DUrole{default_value}{6371.0}}\sphinxparamcomma \sphinxparam{\DUrole{n}{r}\DUrole{o}{=}\DUrole{default_value}{40}}\sphinxparamcomma \sphinxparam{\DUrole{n}{s}\DUrole{o}{=}\DUrole{default_value}{0.6}}}
{}
\pysigstopsignatures
\sphinxAtStartPar
Gaussian Haversine Model
\begin{quote}\begin{description}
\sphinxlineitem{Parameters}\begin{itemize}
\item {} 
\sphinxAtStartPar
\sphinxstyleliteralstrong{\sphinxupquote{df}} (\sphinxstyleliteralemphasis{\sphinxupquote{polars.DataFrame}}) \textendash{} Observations, required columns are “lat” and “lon” representing
latitude and longitude respectively.

\item {} 
\sphinxAtStartPar
\sphinxstyleliteralstrong{\sphinxupquote{R}} (\sphinxstyleliteralemphasis{\sphinxupquote{float}}) \textendash{} Radius of the sphere on which Haversine distance is computed. Defaults
to radius of earth in km.

\item {} 
\sphinxAtStartPar
\sphinxstyleliteralstrong{\sphinxupquote{r}} (\sphinxstyleliteralemphasis{\sphinxupquote{float}}) \textendash{} Gaussian model range parameter

\item {} 
\sphinxAtStartPar
\sphinxstyleliteralstrong{\sphinxupquote{s}} (\sphinxstyleliteralemphasis{\sphinxupquote{float}}) \textendash{} Gaussian model scale parameter

\end{itemize}

\sphinxlineitem{Returns}
\sphinxAtStartPar
\sphinxstylestrong{C} \textendash{} Distance matrix for the input positions. Result has been modified using
the Gaussian model.

\sphinxlineitem{Return type}
\sphinxAtStartPar
np.ndarray

\end{description}\end{quote}

\end{fulllineitems}

\index{radial\_dist() (in module glomar\_gridding.distances)@\spxentry{radial\_dist()}\spxextra{in module glomar\_gridding.distances}}

\begin{fulllineitems}
\phantomsection\label{\detokenize{users_guide:glomar_gridding.distances.radial_dist}}
\pysigstartsignatures
\pysiglinewithargsret
{\sphinxcode{\sphinxupquote{glomar\_gridding.distances.}}\sphinxbfcode{\sphinxupquote{radial\_dist}}}
{\sphinxparam{\DUrole{n}{lat1}}\sphinxparamcomma \sphinxparam{\DUrole{n}{lon1}}\sphinxparamcomma \sphinxparam{\DUrole{n}{lat2}}\sphinxparamcomma \sphinxparam{\DUrole{n}{lon2}}}
{}
\pysigstopsignatures\begin{quote}

\sphinxAtStartPar
Computes a distance matrix of the coordinates using a spherical metric.
\end{quote}
\begin{quote}\begin{description}
\sphinxlineitem{Parameters}\begin{itemize}
\item {} 
\sphinxAtStartPar
\sphinxstyleliteralstrong{\sphinxupquote{lat1}} (\sphinxstyleliteralemphasis{\sphinxupquote{float}}) \textendash{} latitude of point A

\item {} 
\sphinxAtStartPar
\sphinxstyleliteralstrong{\sphinxupquote{lon1}} (\sphinxstyleliteralemphasis{\sphinxupquote{float}}) \textendash{} longitude of point A

\item {} 
\sphinxAtStartPar
\sphinxstyleliteralstrong{\sphinxupquote{lat2}} (\sphinxstyleliteralemphasis{\sphinxupquote{float}}) \textendash{} latitude of point B

\item {} 
\sphinxAtStartPar
\sphinxstyleliteralstrong{\sphinxupquote{lon2}} (\sphinxstyleliteralemphasis{\sphinxupquote{float}}) \textendash{} longitude of point B

\end{itemize}

\sphinxlineitem{Return type}
\sphinxAtStartPar
Radial distance between point A and point B

\end{description}\end{quote}

\end{fulllineitems}

\index{tau\_dist() (in module glomar\_gridding.distances)@\spxentry{tau\_dist()}\spxextra{in module glomar\_gridding.distances}}

\begin{fulllineitems}
\phantomsection\label{\detokenize{users_guide:glomar_gridding.distances.tau_dist}}
\pysigstartsignatures
\pysiglinewithargsret
{\sphinxcode{\sphinxupquote{glomar\_gridding.distances.}}\sphinxbfcode{\sphinxupquote{tau\_dist}}}
{\sphinxparam{\DUrole{n}{df}}}
{}
\pysigstopsignatures
\sphinxAtStartPar
Compute the tau/Mahalanobis matrix for all records within a gridbox

\sphinxAtStartPar
Can be used as an input function for observations.dist\_weight.

\sphinxAtStartPar
Eq.15 in Karspeck paper
but it is standard formulation to the
Mahalanobis distance
\sphinxurl{https://en.wikipedia.org/wiki/Mahalanobis\_distance}
10.1002/qj.900

\sphinxAtStartPar
By Steven Chan \sphinxhyphen{} @stchan
\begin{quote}\begin{description}
\sphinxlineitem{Parameters}
\sphinxAtStartPar
\sphinxstyleliteralstrong{\sphinxupquote{df}} (\sphinxstyleliteralemphasis{\sphinxupquote{polars.DataFrame}}) \textendash{} The observational DataFrame, containing positional information for each
observation (“lat”, “lon”), gridbox specific positional information
(“grid\_lat”, “grid\_lon”), and ellipse length\sphinxhyphen{}scale parameters used for
computation of \sphinxtitleref{sigma} (“grid\_lx”, “grid\_ly”, “grid\_theta”).

\sphinxlineitem{Returns}
\sphinxAtStartPar
\sphinxstylestrong{tau} \textendash{} A matrix of dimension n x n where n is the number of rows in \sphinxtitleref{df} and
is the tau/Mahalanobis distance.

\sphinxlineitem{Return type}
\sphinxAtStartPar
numpy.matrix

\end{description}\end{quote}

\end{fulllineitems}

\index{module@\spxentry{module}!glomar\_gridding.interpolation\_covariance@\spxentry{glomar\_gridding.interpolation\_covariance}}\index{glomar\_gridding.interpolation\_covariance@\spxentry{glomar\_gridding.interpolation\_covariance}!module@\spxentry{module}}

\section{Interpolation Covariance}
\label{\detokenize{users_guide:interpolation-covariance}}\label{\detokenize{users_guide:module-glomar_gridding.interpolation_covariance}}
\sphinxAtStartPar
Functions for computing (components of) the interpolation covariance matrix
used for the interpolation step.
\index{load\_covariance() (in module glomar\_gridding.interpolation\_covariance)@\spxentry{load\_covariance()}\spxextra{in module glomar\_gridding.interpolation\_covariance}}

\begin{fulllineitems}
\phantomsection\label{\detokenize{users_guide:glomar_gridding.interpolation_covariance.load_covariance}}
\pysigstartsignatures
\pysiglinewithargsret
{\sphinxcode{\sphinxupquote{glomar\_gridding.interpolation\_covariance.}}\sphinxbfcode{\sphinxupquote{load\_covariance}}}
{\sphinxparam{\DUrole{n}{path}}\sphinxparamcomma \sphinxparam{\DUrole{n}{cov\_var\_name}\DUrole{o}{=}\DUrole{default_value}{\textquotesingle{}covariance\textquotesingle{}}}\sphinxparamcomma \sphinxparam{\DUrole{o}{**}\DUrole{n}{kwargs}}}
{}
\pysigstopsignatures
\sphinxAtStartPar
Load a covariance matrix from a netCDF file. Can input a filename or a
string to format with keyword arguments.
\begin{quote}\begin{description}
\sphinxlineitem{Parameters}\begin{itemize}
\item {} 
\sphinxAtStartPar
\sphinxstyleliteralstrong{\sphinxupquote{path}} (\sphinxstyleliteralemphasis{\sphinxupquote{str}}) \textendash{} Full filename (including path), or filename with replacements using
str.format with named replacements. For example:
/path/to/global\_covariance\_\{month:02d\}.nc

\item {} 
\sphinxAtStartPar
\sphinxstyleliteralstrong{\sphinxupquote{cov\_var\_name}} (\sphinxstyleliteralemphasis{\sphinxupquote{str}}) \textendash{} Name of the variable for the covariance matrix

\item {} 
\sphinxAtStartPar
\sphinxstyleliteralstrong{\sphinxupquote{**kwargs}} \textendash{} Keywords arguments matching the replacements in the input path.

\end{itemize}

\sphinxlineitem{Returns}
\sphinxAtStartPar
\sphinxstylestrong{covariance} \textendash{} A numpy matrix containing the covariance matrix loaded from the netCDF
file determined by the input arguments.

\sphinxlineitem{Return type}
\sphinxAtStartPar
numpy.ndarray

\end{description}\end{quote}

\end{fulllineitems}

\index{module@\spxentry{module}!glomar\_gridding.error\_covariance@\spxentry{glomar\_gridding.error\_covariance}}\index{glomar\_gridding.error\_covariance@\spxentry{glomar\_gridding.error\_covariance}!module@\spxentry{module}}

\section{Error Covariance}
\label{\detokenize{users_guide:error-covariance}}\label{\detokenize{users_guide:module-glomar_gridding.error_covariance}}
\sphinxAtStartPar
Functions for computing correlated and uncorrelated components of the error
covariance. These values are determined from standard deviation (sigma) values
assigned to groupings within the observational data.

\sphinxAtStartPar
The correlated components will form a matrix that is permutationally equivalent
to a block diagonal matrix (i.e. the matrix will be block diagonal if the
observational data is sorted by the group).

\sphinxAtStartPar
The uncorrelated components will form a diagonal matrix.

\sphinxAtStartPar
Further a distance\sphinxhyphen{}based component can be constructed, where distances between
records within the same grid box are evaluated.

\sphinxAtStartPar
The functions in this module are valid for observational data where there could
be more than 1 observation in a gridbox.
\index{correlated\_components() (in module glomar\_gridding.error\_covariance)@\spxentry{correlated\_components()}\spxextra{in module glomar\_gridding.error\_covariance}}

\begin{fulllineitems}
\phantomsection\label{\detokenize{users_guide:glomar_gridding.error_covariance.correlated_components}}
\pysigstartsignatures
\pysiglinewithargsret
{\sphinxcode{\sphinxupquote{glomar\_gridding.error\_covariance.}}\sphinxbfcode{\sphinxupquote{correlated\_components}}}
{\sphinxparam{\DUrole{n}{df}}\sphinxparamcomma \sphinxparam{\DUrole{n}{group\_col}}\sphinxparamcomma \sphinxparam{\DUrole{n}{bias\_sig\_col}\DUrole{o}{=}\DUrole{default_value}{None}}\sphinxparamcomma \sphinxparam{\DUrole{n}{bias\_sig\_map}\DUrole{o}{=}\DUrole{default_value}{None}}}
{}
\pysigstopsignatures
\sphinxAtStartPar
Returns measurements covariance matrix updated by adding bias uncertainty to
the measurements based on a grouping within the observational data.

\sphinxAtStartPar
The result is equivalent to a block diagonal matrix via permutation. If the
input observational data is sorted by the group column then the resulting
matrix is block diagonal, where the blocks are the size of each grouping.
The values in each block are the square of the sigma value associated with
the grouping.

\sphinxAtStartPar
Note that in most cases the output is not a block\sphinxhyphen{}diagonal, as the input
is not usually sorted by the group column. In most processing cases, the
input dataframe will be sorted by the gridbox index.

\sphinxAtStartPar
The values can either be pre\sphinxhyphen{}defined in the observational dataframe, and
can be indicated by the “bias\_val\_col” argument. Alternatively, a mapping
can be passed, the values will be then assigned by this mapping of group to
sigma.
\begin{quote}\begin{description}
\sphinxlineitem{Parameters}\begin{itemize}
\item {} 
\sphinxAtStartPar
\sphinxstyleliteralstrong{\sphinxupquote{df}} (\sphinxstyleliteralemphasis{\sphinxupquote{polars.DataFrame}}) \textendash{} Observational DataFrame including group information and bias uncertainty
values for each grouping. It is assumed that a single bias uncertainty
value applies to the whole group, and is applied as cross terms in the
covariance matrix (plus to the diagonal).

\item {} 
\sphinxAtStartPar
\sphinxstyleliteralstrong{\sphinxupquote{group\_col}} (\sphinxstyleliteralemphasis{\sphinxupquote{str}}) \textendash{} Name of the column that can be used to partition the observational
DataFrame.

\item {} 
\sphinxAtStartPar
\sphinxstyleliteralstrong{\sphinxupquote{bias\_sig\_col}} (\sphinxstyleliteralemphasis{\sphinxupquote{str}}\sphinxstyleliteralemphasis{\sphinxupquote{ | }}\sphinxstyleliteralemphasis{\sphinxupquote{None}}) \textendash{} Name of the column containing bias uncertainty values for each of
the groups identified by ‘group\_col’. It is assumed that a single bias
uncertainty value applies to the whole group, and is applied as cross
terms in the covariance matrix (plus to the diagonal).

\item {} 
\sphinxAtStartPar
\sphinxstyleliteralstrong{\sphinxupquote{bias\_sig\_map}} (\sphinxstyleliteralemphasis{\sphinxupquote{dict}}\sphinxstyleliteralemphasis{\sphinxupquote{{[}}}\sphinxstyleliteralemphasis{\sphinxupquote{str}}\sphinxstyleliteralemphasis{\sphinxupquote{, }}\sphinxstyleliteralemphasis{\sphinxupquote{float}}\sphinxstyleliteralemphasis{\sphinxupquote{{]} }}\sphinxstyleliteralemphasis{\sphinxupquote{| }}\sphinxstyleliteralemphasis{\sphinxupquote{None}}) \textendash{} Mapping between values in the group\_col and bias uncertainty values,
if bias\_val\_col is not in the DataFrame.

\end{itemize}

\sphinxlineitem{Return type}
\sphinxAtStartPar
The correlated components of the error covariance.

\end{description}\end{quote}

\end{fulllineitems}

\index{dist\_weight() (in module glomar\_gridding.error\_covariance)@\spxentry{dist\_weight()}\spxextra{in module glomar\_gridding.error\_covariance}}

\begin{fulllineitems}
\phantomsection\label{\detokenize{users_guide:glomar_gridding.error_covariance.dist_weight}}
\pysigstartsignatures
\pysiglinewithargsret
{\sphinxcode{\sphinxupquote{glomar\_gridding.error\_covariance.}}\sphinxbfcode{\sphinxupquote{dist\_weight}}}
{\sphinxparam{\DUrole{n}{df}}\sphinxparamcomma \sphinxparam{\DUrole{n}{dist\_fn}}\sphinxparamcomma \sphinxparam{\DUrole{n}{grid\_idx}\DUrole{o}{=}\DUrole{default_value}{\textquotesingle{}grid\_idx\textquotesingle{}}}\sphinxparamcomma \sphinxparam{\DUrole{o}{**}\DUrole{n}{dist\_kwargs}}}
{}
\pysigstopsignatures
\sphinxAtStartPar
Compute the distance and weight matrices over gridboxes for an input Frame.

\sphinxAtStartPar
This function acts as a wrapper for a distance function, allowing for
computation of the distances between positions in the same gridbox using any
distance metric.

\sphinxAtStartPar
The weightings from this function are for the gridbox mean of the
observations within a gridbox.
\begin{quote}\begin{description}
\sphinxlineitem{Parameters}\begin{itemize}
\item {} 
\sphinxAtStartPar
\sphinxstyleliteralstrong{\sphinxupquote{df}} (\sphinxstyleliteralemphasis{\sphinxupquote{polars.DataFrame}}) \textendash{} The observation DataFrame, containing the columns required for
computation of the distance matrix. Contains the “grid\_idx” column which
indicates the gridbox for a given observation. The index of the
DataFrame should match the index ordering for the output distance
matrix/weights.

\item {} 
\sphinxAtStartPar
\sphinxstyleliteralstrong{\sphinxupquote{dist\_fn}} (\sphinxstyleliteralemphasis{\sphinxupquote{Callable}}) \textendash{} 
\sphinxAtStartPar
The function used to compute a distance matrix for all points in a given
grid\sphinxhyphen{}cell. Takes as input a polars.DataFrame as first argument. Any
other arguments should be constant over all gridboxes, or can be a
look\sphinxhyphen{}up table that can use values in the DataFrame to specify values
specific to a gridbox. The function should return a numpy matrix, which
is the distance matrix for the gridbox only. This wrapper function will
correctly apply this matrix to the larger distance matrix using the
index from the DataFrame.

\sphinxAtStartPar
If dist\_fn is None, then no distances are computed and None is returned
for the dist value.


\item {} 
\sphinxAtStartPar
\sphinxstyleliteralstrong{\sphinxupquote{**dist\_kwargs}} \textendash{} Arguments to be passed to dist\_fn. In general these should be constant
across all gridboxes. It is possible to pass a look\sphinxhyphen{}up table that
contains pre\sphinxhyphen{}computed values that are gridbox specific, if the keys can
be matched to a column in df.

\end{itemize}

\sphinxlineitem{Return type}
\sphinxAtStartPar
\DUrole{sphinx_autodoc_typehints-type}{\sphinxcode{\sphinxupquote{tuple}}{[}\sphinxcode{\sphinxupquote{ndarray}}, \sphinxcode{\sphinxupquote{ndarray}}{]}}

\sphinxlineitem{Returns}
\sphinxAtStartPar
\begin{itemize}
\item {} 
\sphinxAtStartPar
\sphinxstylestrong{dist} (\sphinxstyleemphasis{numpy.matrix}) \textendash{} The distance matrix, which contains the same number of rows and columns
as rows in the input DataFrame df. The values in the matrix are 0 if the
indices of the row/column are for observations from different gridboxes,
and non\sphinxhyphen{}zero if the row/column indices fall within the same gridbox.
Consequently, with appropriate re\sphinxhyphen{}arrangement of rows and columns this
matrix can be transformed into a block\sphinxhyphen{}diagonal matrix. If the DataFrame
input is pre\sphinxhyphen{}sorted by the gridbox column, then the result is a
block\sphinxhyphen{}diagonal matrix.

\sphinxAtStartPar
If dist\_fn is None, then this value will be None.

\item {} 
\sphinxAtStartPar
\sphinxstylestrong{weights} (\sphinxstyleemphasis{numpy.matrix}) \textendash{} A matrix of weights. This has dimensions n x p where n is the number of
unique gridboxes and p is the number of observations (the number of rows
in df). The values are 0 if the row and column do not correspond to the
same gridbox and equal to the inverse of the number of observations in a
gridbox if the row and column indices fall within the same gridbox. The
rows of weights are in a sorted order of the gridbox. Should this be
incorrect, one should re\sphinxhyphen{}arrange the rows after calling this function.

\end{itemize}


\end{description}\end{quote}

\end{fulllineitems}

\index{get\_weights() (in module glomar\_gridding.error\_covariance)@\spxentry{get\_weights()}\spxextra{in module glomar\_gridding.error\_covariance}}

\begin{fulllineitems}
\phantomsection\label{\detokenize{users_guide:glomar_gridding.error_covariance.get_weights}}
\pysigstartsignatures
\pysiglinewithargsret
{\sphinxcode{\sphinxupquote{glomar\_gridding.error\_covariance.}}\sphinxbfcode{\sphinxupquote{get\_weights}}}
{\sphinxparam{\DUrole{n}{df}}\sphinxparamcomma \sphinxparam{\DUrole{n}{grid\_idx}\DUrole{o}{=}\DUrole{default_value}{\textquotesingle{}grid\_idx\textquotesingle{}}}}
{}
\pysigstopsignatures
\sphinxAtStartPar
Get just the weight matrices over gridboxes for an input Frame.

\sphinxAtStartPar
The weightings from this function are for the gridbox mean of the
observations within a gridbox.
\begin{quote}\begin{description}
\sphinxlineitem{Parameters}\begin{itemize}
\item {} 
\sphinxAtStartPar
\sphinxstyleliteralstrong{\sphinxupquote{df}} (\sphinxstyleliteralemphasis{\sphinxupquote{polars.DataFrame}}) \textendash{} The observation DataFrame, containing the columns required for
computation of the distance matrix. Contains the “grid\_idx” column which
indicates the gridbox for a given observation. The index of the
DataFrame should match the index ordering for the output weights.

\item {} 
\sphinxAtStartPar
\sphinxstyleliteralstrong{\sphinxupquote{grid\_idx}} (\sphinxstyleliteralemphasis{\sphinxupquote{str}}) \textendash{} Name of the column containing the gridbox index from the output grid.

\end{itemize}

\sphinxlineitem{Returns}
\sphinxAtStartPar
\sphinxstylestrong{weights} \textendash{} A matrix of weights. This has dimensions n x p where n is the number of
unique gridboxes and p is the number of observations (the number of rows
in df). The values are 0 if the row and column do not correspond to the
same gridbox and equal to the inverse of the number of observations in a
gridbox if the row and column indices fall within the same gridbox. The
rows of weights are in a sorted order of the gridbox. Should this be
incorrect, one should re\sphinxhyphen{}arrange the rows after calling this function.

\sphinxlineitem{Return type}
\sphinxAtStartPar
numpy.matrix

\end{description}\end{quote}

\end{fulllineitems}

\index{uncorrelated\_components() (in module glomar\_gridding.error\_covariance)@\spxentry{uncorrelated\_components()}\spxextra{in module glomar\_gridding.error\_covariance}}

\begin{fulllineitems}
\phantomsection\label{\detokenize{users_guide:glomar_gridding.error_covariance.uncorrelated_components}}
\pysigstartsignatures
\pysiglinewithargsret
{\sphinxcode{\sphinxupquote{glomar\_gridding.error\_covariance.}}\sphinxbfcode{\sphinxupquote{uncorrelated\_components}}}
{\sphinxparam{\DUrole{n}{df}}\sphinxparamcomma \sphinxparam{\DUrole{n}{group\_col}\DUrole{o}{=}\DUrole{default_value}{\textquotesingle{}data\_type\textquotesingle{}}}\sphinxparamcomma \sphinxparam{\DUrole{n}{obs\_sig\_col}\DUrole{o}{=}\DUrole{default_value}{None}}\sphinxparamcomma \sphinxparam{\DUrole{n}{obs\_sig\_map}\DUrole{o}{=}\DUrole{default_value}{None}}}
{}
\pysigstopsignatures
\sphinxAtStartPar
Calculates the covariance matrix of the measurements (obervations). This
is the uncorrelated component of the covariance.

\sphinxAtStartPar
The result is a diagonal matrix. The diagonal is formed by the square of the
sigma values associated with the values in the grouping.

\sphinxAtStartPar
The values can either be pre\sphinxhyphen{}defined in the observational dataframe, and
can be indicated by the “bias\_val\_col” argument. Alternatively, a mapping
can be passed, the values will be then assigned by this mapping of group to
sigma.
\begin{quote}\begin{description}
\sphinxlineitem{Parameters}\begin{itemize}
\item {} 
\sphinxAtStartPar
\sphinxstyleliteralstrong{\sphinxupquote{df}} (\sphinxstyleliteralemphasis{\sphinxupquote{polars.DataFrame}}) \textendash{} The observational DataFrame containing values to group by.

\item {} 
\sphinxAtStartPar
\sphinxstyleliteralstrong{\sphinxupquote{group\_col}} (\sphinxstyleliteralemphasis{\sphinxupquote{str}}) \textendash{} Name of the group column to use to set observational sigma values.

\item {} 
\sphinxAtStartPar
\sphinxstyleliteralstrong{\sphinxupquote{obs\_sig\_col}} (\sphinxstyleliteralemphasis{\sphinxupquote{str}}\sphinxstyleliteralemphasis{\sphinxupquote{ | }}\sphinxstyleliteralemphasis{\sphinxupquote{None}}) \textendash{} Name of the column containing observational sigma values. If set and
present in the DataFrame, then this column is used as the diagonal of
the returned covariance matrix.

\item {} 
\sphinxAtStartPar
\sphinxstyleliteralstrong{\sphinxupquote{obs\_sig\_map}} (\sphinxstyleliteralemphasis{\sphinxupquote{dict}}\sphinxstyleliteralemphasis{\sphinxupquote{{[}}}\sphinxstyleliteralemphasis{\sphinxupquote{str}}\sphinxstyleliteralemphasis{\sphinxupquote{, }}\sphinxstyleliteralemphasis{\sphinxupquote{float}}\sphinxstyleliteralemphasis{\sphinxupquote{{]} }}\sphinxstyleliteralemphasis{\sphinxupquote{| }}\sphinxstyleliteralemphasis{\sphinxupquote{None}}) \textendash{} Mapping between group and observational sigma values used to define
the diagonal of the returned covariance matrix.

\end{itemize}

\sphinxlineitem{Return type}
\sphinxAtStartPar
\DUrole{sphinx_autodoc_typehints-type}{\sphinxcode{\sphinxupquote{ndarray}}}

\sphinxlineitem{Returns}
\sphinxAtStartPar
\begin{itemize}
\item {} 
\sphinxAtStartPar
\sphinxstyleemphasis{A diagonal matrix representing the uncorrelated components of the error}

\item {} 
\sphinxAtStartPar
\sphinxstyleemphasis{covariance matrix.}

\end{itemize}


\end{description}\end{quote}

\end{fulllineitems}

\index{module@\spxentry{module}!glomar\_gridding.perturbation@\spxentry{glomar\_gridding.perturbation}}\index{glomar\_gridding.perturbation@\spxentry{glomar\_gridding.perturbation}!module@\spxentry{module}}\phantomsection\label{\detokenize{users_guide:module-glomar_gridding.perturbation}}
\sphinxAtStartPar
Functions for helping with perturbations/random drawing
\index{scipy\_mv\_normal\_draw() (in module glomar\_gridding.perturbation)@\spxentry{scipy\_mv\_normal\_draw()}\spxextra{in module glomar\_gridding.perturbation}}

\begin{fulllineitems}
\phantomsection\label{\detokenize{users_guide:glomar_gridding.perturbation.scipy_mv_normal_draw}}
\pysigstartsignatures
\pysiglinewithargsret
{\sphinxcode{\sphinxupquote{glomar\_gridding.perturbation.}}\sphinxbfcode{\sphinxupquote{scipy\_mv\_normal\_draw}}}
{\sphinxparam{\DUrole{n}{loc}}\sphinxparamcomma \sphinxparam{\DUrole{n}{cov}}\sphinxparamcomma \sphinxparam{\DUrole{n}{ndraws}\DUrole{o}{=}\DUrole{default_value}{1}}\sphinxparamcomma \sphinxparam{\DUrole{n}{eigen\_rtol}\DUrole{o}{=}\DUrole{default_value}{1e\sphinxhyphen{}06}}\sphinxparamcomma \sphinxparam{\DUrole{n}{eigen\_fudge}\DUrole{o}{=}\DUrole{default_value}{1e\sphinxhyphen{}08}}}
{}
\pysigstopsignatures
\sphinxAtStartPar
Do a random multivariate normal draw using
scipy.stats.multivariate\_normal.rvs

\sphinxAtStartPar
numpy.random.multivariate\_normal can also,
but fixing seeds are more difficult using numpy

\sphinxAtStartPar
This function has similar API as GP\_draw with less kwargs.

\sphinxAtStartPar
Warning/possible future scipy version may change this:
It seems if one uses stats.Covariance, you have to have add {[}0{]} from rvs
function. The above behavior applies to scipy v1.14.0
\begin{quote}\begin{description}
\sphinxlineitem{Parameters}\begin{itemize}
\item {} 
\sphinxAtStartPar
\sphinxstyleliteralstrong{\sphinxupquote{loc}} (\sphinxstyleliteralemphasis{\sphinxupquote{float}}) \textendash{} the location for the normal dry

\item {} 
\sphinxAtStartPar
\sphinxstyleliteralstrong{\sphinxupquote{cov}} (\sphinxstyleliteralemphasis{\sphinxupquote{numpy.ndarray}}) \textendash{} not a xarray/iris cube! Some of our covariances are saved in numpy
format and not netCDF files

\item {} 
\sphinxAtStartPar
\sphinxstyleliteralstrong{\sphinxupquote{n\_draws}} (\sphinxstyleliteralemphasis{\sphinxupquote{int}}) \textendash{} number of simulations, this is usually set to 1 except during

\item {} 
\sphinxAtStartPar
\sphinxstyleliteralstrong{\sphinxupquote{testing}} (\sphinxstyleliteralemphasis{\sphinxupquote{unit}})

\item {} 
\sphinxAtStartPar
\sphinxstyleliteralstrong{\sphinxupquote{eigen\_rtol}} (\sphinxstyleliteralemphasis{\sphinxupquote{float}}) \textendash{} relative tolerance to negative eigenvalues

\item {} 
\sphinxAtStartPar
\sphinxstyleliteralstrong{\sphinxupquote{eigen\_fudge}} (\sphinxstyleliteralemphasis{\sphinxupquote{float}}) \textendash{} forced minimum value of eigenvalues if negative values are detected

\end{itemize}

\sphinxlineitem{Returns}
\sphinxAtStartPar
\sphinxstylestrong{draw} \textendash{} The draw(s) from the multivariate random normal distribution defined
by the loc and cov parameters. If the cov parameter is not
positive\sphinxhyphen{}definite then a new covariance will be determined by adjusting
the eigen decomposition such that the modified covariance should be
positive\sphinxhyphen{}definite.

\sphinxlineitem{Return type}
\sphinxAtStartPar
np.ndarray

\end{description}\end{quote}

\end{fulllineitems}

\index{module@\spxentry{module}!glomar\_gridding.utils@\spxentry{glomar\_gridding.utils}}\index{glomar\_gridding.utils@\spxentry{glomar\_gridding.utils}!module@\spxentry{module}}

\section{Utils}
\label{\detokenize{users_guide:utils}}\label{\detokenize{users_guide:module-glomar_gridding.utils}}
\sphinxAtStartPar
Utility functions for GlomarGridding
\index{ColumnNotFoundError@\spxentry{ColumnNotFoundError}}

\begin{fulllineitems}
\phantomsection\label{\detokenize{users_guide:glomar_gridding.utils.ColumnNotFoundError}}
\pysigstartsignatures
\pysigline
{\sphinxbfcode{\sphinxupquote{\DUrole{k}{exception}\DUrole{w}{ }}}\sphinxcode{\sphinxupquote{glomar\_gridding.utils.}}\sphinxbfcode{\sphinxupquote{ColumnNotFoundError}}}
\pysigstopsignatures
\sphinxAtStartPar
Error class for Column Not Being Found

\end{fulllineitems}

\index{ConfigParserMultiValues (class in glomar\_gridding.utils)@\spxentry{ConfigParserMultiValues}\spxextra{class in glomar\_gridding.utils}}

\begin{fulllineitems}
\phantomsection\label{\detokenize{users_guide:glomar_gridding.utils.ConfigParserMultiValues}}
\pysigstartsignatures
\pysigline
{\sphinxbfcode{\sphinxupquote{\DUrole{k}{class}\DUrole{w}{ }}}\sphinxcode{\sphinxupquote{glomar\_gridding.utils.}}\sphinxbfcode{\sphinxupquote{ConfigParserMultiValues}}}
\pysigstopsignatures
\sphinxAtStartPar
Internal Helper Class

\end{fulllineitems}

\index{MonthName (class in glomar\_gridding.utils)@\spxentry{MonthName}\spxextra{class in glomar\_gridding.utils}}

\begin{fulllineitems}
\phantomsection\label{\detokenize{users_guide:glomar_gridding.utils.MonthName}}
\pysigstartsignatures
\pysiglinewithargsret
{\sphinxbfcode{\sphinxupquote{\DUrole{k}{class}\DUrole{w}{ }}}\sphinxcode{\sphinxupquote{glomar\_gridding.utils.}}\sphinxbfcode{\sphinxupquote{MonthName}}}
{\sphinxparam{\DUrole{n}{value}}}
{}
\pysigstopsignatures
\sphinxAtStartPar
Name of month from int

\end{fulllineitems}

\index{add\_empty\_layers() (in module glomar\_gridding.utils)@\spxentry{add\_empty\_layers()}\spxextra{in module glomar\_gridding.utils}}

\begin{fulllineitems}
\phantomsection\label{\detokenize{users_guide:glomar_gridding.utils.add_empty_layers}}
\pysigstartsignatures
\pysiglinewithargsret
{\sphinxcode{\sphinxupquote{glomar\_gridding.utils.}}\sphinxbfcode{\sphinxupquote{add\_empty\_layers}}}
{\sphinxparam{\DUrole{n}{nc\_variables}}\sphinxparamcomma \sphinxparam{\DUrole{n}{timestamps}}\sphinxparamcomma \sphinxparam{\DUrole{n}{shape}}}
{}
\pysigstopsignatures
\sphinxAtStartPar
Add empty layers to a netcdf file. This adds a layer of zeros to the netCDF
file.
\begin{quote}\begin{description}
\sphinxlineitem{Parameters}\begin{itemize}
\item {} 
\sphinxAtStartPar
\sphinxstyleliteralstrong{\sphinxupquote{nc\_variables}} (\sphinxstyleliteralemphasis{\sphinxupquote{Iterable}}\sphinxstyleliteralemphasis{\sphinxupquote{{[}}}\sphinxstyleliteralemphasis{\sphinxupquote{nc.Variable}}\sphinxstyleliteralemphasis{\sphinxupquote{{]} }}\sphinxstyleliteralemphasis{\sphinxupquote{| }}\sphinxstyleliteralemphasis{\sphinxupquote{nc.Variable}}) \textendash{} Name(s) of the variables to add empty layers to

\item {} 
\sphinxAtStartPar
\sphinxstyleliteralstrong{\sphinxupquote{timestamps}} (\sphinxstyleliteralemphasis{\sphinxupquote{Iterable}}\sphinxstyleliteralemphasis{\sphinxupquote{{[}}}\sphinxstyleliteralemphasis{\sphinxupquote{int}}\sphinxstyleliteralemphasis{\sphinxupquote{{]} }}\sphinxstyleliteralemphasis{\sphinxupquote{| }}\sphinxstyleliteralemphasis{\sphinxupquote{int}}) \textendash{} Indices to add empty layers

\item {} 
\sphinxAtStartPar
\sphinxstyleliteralstrong{\sphinxupquote{shape}} (\sphinxstyleliteralemphasis{\sphinxupquote{tuple}}\sphinxstyleliteralemphasis{\sphinxupquote{{[}}}\sphinxstyleliteralemphasis{\sphinxupquote{int}}\sphinxstyleliteralemphasis{\sphinxupquote{, }}\sphinxstyleliteralemphasis{\sphinxupquote{int}}\sphinxstyleliteralemphasis{\sphinxupquote{{]}}}) \textendash{} Shape of the layer to add

\end{itemize}

\sphinxlineitem{Return type}
\sphinxAtStartPar
\DUrole{sphinx_autodoc_typehints-type}{\sphinxcode{\sphinxupquote{None}}}

\end{description}\end{quote}

\end{fulllineitems}

\index{adjust\_small\_negative() (in module glomar\_gridding.utils)@\spxentry{adjust\_small\_negative()}\spxextra{in module glomar\_gridding.utils}}

\begin{fulllineitems}
\phantomsection\label{\detokenize{users_guide:glomar_gridding.utils.adjust_small_negative}}
\pysigstartsignatures
\pysiglinewithargsret
{\sphinxcode{\sphinxupquote{glomar\_gridding.utils.}}\sphinxbfcode{\sphinxupquote{adjust\_small\_negative}}}
{\sphinxparam{\DUrole{n}{mat}}}
{}
\pysigstopsignatures
\sphinxAtStartPar
Adjusts small negative values (with absolute value \textless{} 1e\sphinxhyphen{}8)
in matrix to 0 in\sphinxhyphen{}place.

\sphinxAtStartPar
Raises a warning if any small negative values are detected.
\begin{quote}\begin{description}
\sphinxlineitem{Parameters}
\sphinxAtStartPar
\sphinxstyleliteralstrong{\sphinxupquote{mat}} (\sphinxstyleliteralemphasis{\sphinxupquote{np.ndarray}}\sphinxstyleliteralemphasis{\sphinxupquote{{[}}}\sphinxstyleliteralemphasis{\sphinxupquote{float}}\sphinxstyleliteralemphasis{\sphinxupquote{{]}}}) \textendash{} Squared uncertainty associated with chosen kriging method
Derived from the diagonal of the matrix

\sphinxlineitem{Return type}
\sphinxAtStartPar
\DUrole{sphinx_autodoc_typehints-type}{\sphinxcode{\sphinxupquote{ndarray}}}

\end{description}\end{quote}

\end{fulllineitems}

\index{check\_cols() (in module glomar\_gridding.utils)@\spxentry{check\_cols()}\spxextra{in module glomar\_gridding.utils}}

\begin{fulllineitems}
\phantomsection\label{\detokenize{users_guide:glomar_gridding.utils.check_cols}}
\pysigstartsignatures
\pysiglinewithargsret
{\sphinxcode{\sphinxupquote{glomar\_gridding.utils.}}\sphinxbfcode{\sphinxupquote{check\_cols}}}
{\sphinxparam{\DUrole{n}{df}}\sphinxparamcomma \sphinxparam{\DUrole{n}{cols}}}
{}
\pysigstopsignatures
\sphinxAtStartPar
Check that all columns in a list of columns are in a DataFrame
\begin{quote}\begin{description}
\sphinxlineitem{Return type}
\sphinxAtStartPar
\DUrole{sphinx_autodoc_typehints-type}{\sphinxcode{\sphinxupquote{None}}}

\end{description}\end{quote}

\end{fulllineitems}

\index{days\_since\_by\_month() (in module glomar\_gridding.utils)@\spxentry{days\_since\_by\_month()}\spxextra{in module glomar\_gridding.utils}}

\begin{fulllineitems}
\phantomsection\label{\detokenize{users_guide:glomar_gridding.utils.days_since_by_month}}
\pysigstartsignatures
\pysiglinewithargsret
{\sphinxcode{\sphinxupquote{glomar\_gridding.utils.}}\sphinxbfcode{\sphinxupquote{days\_since\_by\_month}}}
{\sphinxparam{\DUrole{n}{year}}\sphinxparamcomma \sphinxparam{\DUrole{n}{day}}}
{}
\pysigstopsignatures
\sphinxAtStartPar
Get the number of days since \sphinxtitleref{year}\sphinxhyphen{}01\sphinxhyphen{}\sphinxtitleref{day} for each month. This is used
to set the time values in a netCDF file where temporal resolution is monthly
and the units are days since some date.
\begin{quote}\begin{description}
\sphinxlineitem{Return type}
\sphinxAtStartPar
\DUrole{sphinx_autodoc_typehints-type}{\sphinxcode{\sphinxupquote{ndarray}}}

\end{description}\end{quote}

\end{fulllineitems}

\index{filter\_bounds() (in module glomar\_gridding.utils)@\spxentry{filter\_bounds()}\spxextra{in module glomar\_gridding.utils}}

\begin{fulllineitems}
\phantomsection\label{\detokenize{users_guide:glomar_gridding.utils.filter_bounds}}
\pysigstartsignatures
\pysiglinewithargsret
{\sphinxcode{\sphinxupquote{glomar\_gridding.utils.}}\sphinxbfcode{\sphinxupquote{filter\_bounds}}}
{\sphinxparam{\DUrole{n}{df}}\sphinxparamcomma \sphinxparam{\DUrole{n}{bounds}}\sphinxparamcomma \sphinxparam{\DUrole{n}{bound\_cols}}\sphinxparamcomma \sphinxparam{\DUrole{n}{closed}\DUrole{o}{=}\DUrole{default_value}{\textquotesingle{}left\textquotesingle{}}}}
{}
\pysigstopsignatures
\sphinxAtStartPar
Filter a polars DataFrame based on a set of lower and upper bounds.
\begin{quote}\begin{description}
\sphinxlineitem{Parameters}\begin{itemize}
\item {} 
\sphinxAtStartPar
\sphinxstyleliteralstrong{\sphinxupquote{df}} (\sphinxstyleliteralemphasis{\sphinxupquote{polars.DataFrame}}) \textendash{} The data to be filtered by the bounds

\item {} 
\sphinxAtStartPar
\sphinxstyleliteralstrong{\sphinxupquote{bounds}} (\sphinxstyleliteralemphasis{\sphinxupquote{list}}\sphinxstyleliteralemphasis{\sphinxupquote{{[}}}\sphinxstyleliteralemphasis{\sphinxupquote{tuple}}\sphinxstyleliteralemphasis{\sphinxupquote{{[}}}\sphinxstyleliteralemphasis{\sphinxupquote{float}}\sphinxstyleliteralemphasis{\sphinxupquote{, }}\sphinxstyleliteralemphasis{\sphinxupquote{float}}\sphinxstyleliteralemphasis{\sphinxupquote{{]}}}\sphinxstyleliteralemphasis{\sphinxupquote{{]}}}) \textendash{} A list of tuples containing lower and upper bounds for a column

\item {} 
\sphinxAtStartPar
\sphinxstyleliteralstrong{\sphinxupquote{bound\_cols}} (\sphinxstyleliteralemphasis{\sphinxupquote{list}}\sphinxstyleliteralemphasis{\sphinxupquote{{[}}}\sphinxstyleliteralemphasis{\sphinxupquote{str}}\sphinxstyleliteralemphasis{\sphinxupquote{{]}}}) \textendash{} A list of column names to be filtered by the bounds, the length of
the bounds list must equal the length of the bound\_cols list.

\item {} 
\sphinxAtStartPar
\sphinxstyleliteralstrong{\sphinxupquote{closed}} (\sphinxstyleliteralemphasis{\sphinxupquote{str}}\sphinxstyleliteralemphasis{\sphinxupquote{ | }}\sphinxstyleliteralemphasis{\sphinxupquote{list}}\sphinxstyleliteralemphasis{\sphinxupquote{{[}}}\sphinxstyleliteralemphasis{\sphinxupquote{str}}\sphinxstyleliteralemphasis{\sphinxupquote{{]}}}) \textendash{} One of “both”, “left”, “right”, “none” indicating the closedness of
the bounds. If the input is a single instance then all bounds will have
that closedness. If it is a list of closed values then its length must
match the length of the bounds list.

\end{itemize}

\sphinxlineitem{Return type}
\sphinxAtStartPar
\DUrole{sphinx_autodoc_typehints-type}{\sphinxcode{\sphinxupquote{DataFrame}}}

\end{description}\end{quote}

\end{fulllineitems}

\index{find\_nearest() (in module glomar\_gridding.utils)@\spxentry{find\_nearest()}\spxextra{in module glomar\_gridding.utils}}

\begin{fulllineitems}
\phantomsection\label{\detokenize{users_guide:glomar_gridding.utils.find_nearest}}
\pysigstartsignatures
\pysiglinewithargsret
{\sphinxcode{\sphinxupquote{glomar\_gridding.utils.}}\sphinxbfcode{\sphinxupquote{find\_nearest}}}
{\sphinxparam{\DUrole{n}{array}}\sphinxparamcomma \sphinxparam{\DUrole{n}{values}}}
{}
\pysigstopsignatures
\sphinxAtStartPar
Get the indices and values from an array that are closest to the input
values.

\sphinxAtStartPar
A single index, value pair is returned for each look\sphinxhyphen{}up value in the values
list.
\begin{quote}\begin{description}
\sphinxlineitem{Parameters}\begin{itemize}
\item {} 
\sphinxAtStartPar
\sphinxstyleliteralstrong{\sphinxupquote{array}} (\sphinxstyleliteralemphasis{\sphinxupquote{Iterable}}) \textendash{} The array to search for nearest values.

\item {} 
\sphinxAtStartPar
\sphinxstyleliteralstrong{\sphinxupquote{values}} (\sphinxstyleliteralemphasis{\sphinxupquote{Iterable}}) \textendash{} The values to look\sphinxhyphen{}up in the array.

\end{itemize}

\sphinxlineitem{Return type}
\sphinxAtStartPar
\DUrole{sphinx_autodoc_typehints-type}{\sphinxcode{\sphinxupquote{tuple}}{[}\sphinxcode{\sphinxupquote{list}}{[}\sphinxcode{\sphinxupquote{int}}{]}, \sphinxcode{\sphinxupquote{ndarray}}{]}}

\sphinxlineitem{Returns}
\sphinxAtStartPar
\begin{itemize}
\item {} 
\sphinxAtStartPar
\sphinxstylestrong{idx\_list} (\sphinxstyleemphasis{list{[}int{]}}) \textendash{} The indices of nearest values

\item {} 
\sphinxAtStartPar
\sphinxstylestrong{array\_values\_list} (\sphinxstyleemphasis{list}) \textendash{} The list of values in array that are closest to the input values.

\end{itemize}


\end{description}\end{quote}

\end{fulllineitems}

\index{get\_date\_index() (in module glomar\_gridding.utils)@\spxentry{get\_date\_index()}\spxextra{in module glomar\_gridding.utils}}

\begin{fulllineitems}
\phantomsection\label{\detokenize{users_guide:glomar_gridding.utils.get_date_index}}
\pysigstartsignatures
\pysiglinewithargsret
{\sphinxcode{\sphinxupquote{glomar\_gridding.utils.}}\sphinxbfcode{\sphinxupquote{get\_date\_index}}}
{\sphinxparam{\DUrole{n}{year}}\sphinxparamcomma \sphinxparam{\DUrole{n}{month}}\sphinxparamcomma \sphinxparam{\DUrole{n}{start\_year}}}
{}
\pysigstopsignatures
\sphinxAtStartPar
Get the index of a given year\sphinxhyphen{}month in a monthly sequence of dates
starting from month 1 in a specific start year
\begin{quote}\begin{description}
\sphinxlineitem{Parameters}\begin{itemize}
\item {} 
\sphinxAtStartPar
\sphinxstyleliteralstrong{\sphinxupquote{year}} (\sphinxstyleliteralemphasis{\sphinxupquote{int}}) \textendash{} The year for the date to find the index of.

\item {} 
\sphinxAtStartPar
\sphinxstyleliteralstrong{\sphinxupquote{month}} (\sphinxstyleliteralemphasis{\sphinxupquote{int}}) \textendash{} The month for the date to find the index of.

\item {} 
\sphinxAtStartPar
\sphinxstyleliteralstrong{\sphinxupquote{start\_year}} (\sphinxstyleliteralemphasis{\sphinxupquote{int}}) \textendash{} The start year of the date series, the result assumes that the date
time series starts in the first month of this year.

\end{itemize}

\sphinxlineitem{Returns}
\sphinxAtStartPar
\sphinxstylestrong{index} \textendash{} The index of the input date in the monthly datetime series starting from
the first month of year \sphinxtitleref{start\_year}.

\sphinxlineitem{Return type}
\sphinxAtStartPar
int

\end{description}\end{quote}

\end{fulllineitems}

\index{get\_pentad\_range() (in module glomar\_gridding.utils)@\spxentry{get\_pentad\_range()}\spxextra{in module glomar\_gridding.utils}}

\begin{fulllineitems}
\phantomsection\label{\detokenize{users_guide:glomar_gridding.utils.get_pentad_range}}
\pysigstartsignatures
\pysiglinewithargsret
{\sphinxcode{\sphinxupquote{glomar\_gridding.utils.}}\sphinxbfcode{\sphinxupquote{get\_pentad\_range}}}
{\sphinxparam{\DUrole{n}{centre\_date}}}
{}
\pysigstopsignatures
\sphinxAtStartPar
Get the start and date of a pentad centred at a centre date. If the
pentad includes the leap date of 29th Feb then the pentad will include
6 days. This follows the \sphinxstylestrong{*} pentad convention.

\sphinxAtStartPar
The start and end date are first calculated from a non\sphinxhyphen{}leap year.

\sphinxAtStartPar
If the centre date value is 29th Feb then the pentad will be a pentad
starting on 27th Feb and ending on 2nd March.
\begin{quote}\begin{description}
\sphinxlineitem{Parameters}
\sphinxAtStartPar
\sphinxstyleliteralstrong{\sphinxupquote{centre\_date}} (\sphinxstyleliteralemphasis{\sphinxupquote{datetime.date}}) \textendash{} The centre date of the pentad. The start date will be 2 days before this
date, and the end date will be 2 days after.

\sphinxlineitem{Return type}
\sphinxAtStartPar
\DUrole{sphinx_autodoc_typehints-type}{\sphinxcode{\sphinxupquote{tuple}}{[}\sphinxcode{\sphinxupquote{date}}, \sphinxcode{\sphinxupquote{date}}{]}}

\sphinxlineitem{Returns}
\sphinxAtStartPar
\begin{itemize}
\item {} 
\sphinxAtStartPar
\sphinxstylestrong{start\_date} (\sphinxstyleemphasis{datetime.date}) \textendash{} Two days before centre\_date

\item {} 
\sphinxAtStartPar
\sphinxstylestrong{end\_date} (\sphinxstyleemphasis{datetime.date}) \textendash{} Two days after centre\_date

\end{itemize}


\end{description}\end{quote}

\end{fulllineitems}

\index{init\_logging() (in module glomar\_gridding.utils)@\spxentry{init\_logging()}\spxextra{in module glomar\_gridding.utils}}

\begin{fulllineitems}
\phantomsection\label{\detokenize{users_guide:glomar_gridding.utils.init_logging}}
\pysigstartsignatures
\pysiglinewithargsret
{\sphinxcode{\sphinxupquote{glomar\_gridding.utils.}}\sphinxbfcode{\sphinxupquote{init\_logging}}}
{\sphinxparam{\DUrole{n}{file}\DUrole{o}{=}\DUrole{default_value}{None}}}
{}
\pysigstopsignatures
\sphinxAtStartPar
Initialise the logger
\begin{quote}\begin{description}
\sphinxlineitem{Parameters}
\sphinxAtStartPar
\sphinxstyleliteralstrong{\sphinxupquote{file}} (\sphinxstyleliteralemphasis{\sphinxupquote{str}}\sphinxstyleliteralemphasis{\sphinxupquote{ | }}\sphinxstyleliteralemphasis{\sphinxupquote{None}}) \textendash{} File to send log messages to. If set to None (default) then print log
messages to STDout

\sphinxlineitem{Return type}
\sphinxAtStartPar
\DUrole{sphinx_autodoc_typehints-type}{\sphinxcode{\sphinxupquote{None}}}

\end{description}\end{quote}

\end{fulllineitems}

\index{intersect\_mtlb() (in module glomar\_gridding.utils)@\spxentry{intersect\_mtlb()}\spxextra{in module glomar\_gridding.utils}}

\begin{fulllineitems}
\phantomsection\label{\detokenize{users_guide:glomar_gridding.utils.intersect_mtlb}}
\pysigstartsignatures
\pysiglinewithargsret
{\sphinxcode{\sphinxupquote{glomar\_gridding.utils.}}\sphinxbfcode{\sphinxupquote{intersect\_mtlb}}}
{\sphinxparam{\DUrole{n}{a}}\sphinxparamcomma \sphinxparam{\DUrole{n}{b}}}
{}
\pysigstopsignatures
\sphinxAtStartPar
Returns data common between two arrays, a and b, in a sorted order and index
vectors for a and b arrays Reproduces behaviour of Matlab’s intersect
function.
\begin{quote}\begin{description}
\sphinxlineitem{Parameters}\begin{itemize}
\item {} 
\sphinxAtStartPar
\sphinxstyleliteralstrong{\sphinxupquote{array}} (\sphinxstyleliteralemphasis{\sphinxupquote{b}}\sphinxstyleliteralemphasis{\sphinxupquote{ (}}\sphinxstyleliteralemphasis{\sphinxupquote{array}}\sphinxstyleliteralemphasis{\sphinxupquote{) }}\sphinxstyleliteralemphasis{\sphinxupquote{\sphinxhyphen{} 1\sphinxhyphen{}D}})

\item {} 
\sphinxAtStartPar
\sphinxstyleliteralstrong{\sphinxupquote{array}}

\end{itemize}

\sphinxlineitem{Returns}
\sphinxAtStartPar
\begin{itemize}
\item {} 
\sphinxAtStartPar
\sphinxstyleemphasis{1\sphinxhyphen{}D array, c, of common values found in two arrays, a and b, sorted in order}

\item {} 
\sphinxAtStartPar
\sphinxstyleemphasis{List of indices, where the common values are located, for array a}

\item {} 
\sphinxAtStartPar
\sphinxstyleemphasis{List of indices, where the common values are located, for array b}

\end{itemize}


\end{description}\end{quote}

\end{fulllineitems}

\index{select\_bounds() (in module glomar\_gridding.utils)@\spxentry{select\_bounds()}\spxextra{in module glomar\_gridding.utils}}

\begin{fulllineitems}
\phantomsection\label{\detokenize{users_guide:glomar_gridding.utils.select_bounds}}
\pysigstartsignatures
\pysiglinewithargsret
{\sphinxcode{\sphinxupquote{glomar\_gridding.utils.}}\sphinxbfcode{\sphinxupquote{select\_bounds}}}
{\sphinxparam{\DUrole{n}{x}}\sphinxparamcomma \sphinxparam{\DUrole{n}{bounds}\DUrole{o}{=}\DUrole{default_value}{{[}(\sphinxhyphen{}90, 90), (\sphinxhyphen{}180, 180){]}}}\sphinxparamcomma \sphinxparam{\DUrole{n}{variables}\DUrole{o}{=}\DUrole{default_value}{{[}\textquotesingle{}lat\textquotesingle{}, \textquotesingle{}lon\textquotesingle{}{]}}}}
{}
\pysigstopsignatures
\sphinxAtStartPar
Filter an xarray.DataArray or xarray.Dataset by a set of bounds.
\begin{quote}\begin{description}
\sphinxlineitem{Parameters}\begin{itemize}
\item {} 
\sphinxAtStartPar
\sphinxstyleliteralstrong{\sphinxupquote{x}} (\sphinxstyleliteralemphasis{\sphinxupquote{xarray.DataArray}}\sphinxstyleliteralemphasis{\sphinxupquote{ | }}\sphinxstyleliteralemphasis{\sphinxupquote{xarray.Dataset}}) \textendash{} The data to filter

\item {} 
\sphinxAtStartPar
\sphinxstyleliteralstrong{\sphinxupquote{bounds}} (\sphinxstyleliteralemphasis{\sphinxupquote{list}}\sphinxstyleliteralemphasis{\sphinxupquote{{[}}}\sphinxstyleliteralemphasis{\sphinxupquote{tuple}}\sphinxstyleliteralemphasis{\sphinxupquote{{[}}}\sphinxstyleliteralemphasis{\sphinxupquote{float}}\sphinxstyleliteralemphasis{\sphinxupquote{, }}\sphinxstyleliteralemphasis{\sphinxupquote{float}}\sphinxstyleliteralemphasis{\sphinxupquote{{]}}}\sphinxstyleliteralemphasis{\sphinxupquote{{]}}}) \textendash{} A list of tuples containing the lower and upper bounds for each
dimension.

\item {} 
\sphinxAtStartPar
\sphinxstyleliteralstrong{\sphinxupquote{variables}} (\sphinxstyleliteralemphasis{\sphinxupquote{list}}\sphinxstyleliteralemphasis{\sphinxupquote{{[}}}\sphinxstyleliteralemphasis{\sphinxupquote{str}}\sphinxstyleliteralemphasis{\sphinxupquote{{]}}}) \textendash{} Names of the dimensions (the order must match the bounds).

\end{itemize}

\sphinxlineitem{Returns}
\sphinxAtStartPar
\sphinxstylestrong{x} \textendash{} The input data filtered by the bounds.

\sphinxlineitem{Return type}
\sphinxAtStartPar
xarray.DataArray | xarray.Dataset

\end{description}\end{quote}

\end{fulllineitems}



\renewcommand{\indexname}{Python Module Index}
\begin{sphinxtheindex}
\let\bigletter\sphinxstyleindexlettergroup
\bigletter{g}
\item\relax\sphinxstyleindexentry{glomar\_gridding.climatology}\sphinxstyleindexpageref{users_guide:\detokenize{module-glomar_gridding.climatology}}
\item\relax\sphinxstyleindexentry{glomar\_gridding.distances}\sphinxstyleindexpageref{users_guide:\detokenize{module-glomar_gridding.distances}}
\item\relax\sphinxstyleindexentry{glomar\_gridding.error\_covariance}\sphinxstyleindexpageref{users_guide:\detokenize{module-glomar_gridding.error_covariance}}
\item\relax\sphinxstyleindexentry{glomar\_gridding.grid}\sphinxstyleindexpageref{users_guide:\detokenize{module-glomar_gridding.grid}}
\item\relax\sphinxstyleindexentry{glomar\_gridding.interpolation\_covariance}\sphinxstyleindexpageref{users_guide:\detokenize{module-glomar_gridding.interpolation_covariance}}
\item\relax\sphinxstyleindexentry{glomar\_gridding.kriging}\sphinxstyleindexpageref{users_guide:\detokenize{module-glomar_gridding.kriging}}
\item\relax\sphinxstyleindexentry{glomar\_gridding.mask}\sphinxstyleindexpageref{users_guide:\detokenize{module-glomar_gridding.mask}}
\item\relax\sphinxstyleindexentry{glomar\_gridding.perturbation}\sphinxstyleindexpageref{users_guide:\detokenize{module-glomar_gridding.perturbation}}
\item\relax\sphinxstyleindexentry{glomar\_gridding.utils}\sphinxstyleindexpageref{users_guide:\detokenize{module-glomar_gridding.utils}}
\item\relax\sphinxstyleindexentry{glomar\_gridding.variogram}\sphinxstyleindexpageref{users_guide:\detokenize{module-glomar_gridding.variogram}}
\end{sphinxtheindex}

\renewcommand{\indexname}{Index}
\printindex
\end{document}